%%%%%%%%%%%%%%%%%%%%%%%%%%%%%%%%%%%%%%%%%%%%%%%%%%%%%%%%%%%%%%%%%%%%%%%%%%%%%%%%%%%%%%%%%%%%%%%%%%%
%% 
%% Stile Latex per la stesura della tesi di laurea presso il
%% Dipartimento di Tecnologie dell'Informazione di Crema
%%
%% Opzioni disponibili da utilizzare in \documentclass:
%%
%% correlatore
%% 		l'opzione permette di visualizzare, se presente il correlatore
%% 		della tesi
%%
%% linea
%%		permette la visualizzazione di una sottile linea per header e
%%		e footer
%%
%% [ informatica | sicurezza | ts | magistrale ] (informatica e' il default value)
%%		imposta il corso di laurea con la corretta dicitura
%%		informatica: Corso di Laurea in Informatica (Crema)
%%		sicurezza: Corso di Laurea in Sicurezza dei Sistemi e delle Reti Informatiche
%% 		ts: Corso di Laurea in Tecnologie per la Societ\`a dell'Informazione
%% 		magistrale: Corso di Laurea in Scienze e Tecnologie dell'Informazione
%%
%% titoliSinistra
%%		con l'impostazione fronteretro la prima pagina di un capitolo e' dispari;
%%		in fase di rilegatura il titolo del capitolo risulterebbe a destra e quindi all'interno;
%%		con questa opzione si aumenta la visibilita' del titolo
%%
%% [ righeNumerate | righeNumeratePerPagina ]
%%		l'opzione permette di inserire un numerino progressivo su ogni riga di testo
%% 		per permettere una piu' facile indicazione in caso di invio delle correzioni in
%%		forma elettronica
%%
%% prefazione
%%		l'opzione permette l'inclusione di una prefazione il cui testo deve essere 
%%		obbligatoriamente contenuta nel file prefazione.tex
%%
%% [ listaFigureInizio | listaFigureFine ]
%%		l'opzione permette l'inclusione della lista delle figure prima oppure dopo
%% 		il testo della tesi; di default, nessuna lista delle figure e' inclusa
%%
%% [ listaTabelleInizio | listaTabelleFine ]
%%		l'opzione permette l'inclusione della lista delle tabelle prima oppure dopo
%% 		il testo della tesi; di default, nessuna lista delle tabelle e' inclusa
%%
%%%%%%%%%%%%%%%%%%%%%%%%%%%%%%%%%%%%%%%%%%%%%%%%%%%%%%%%%%%%%%%%%%%%%%%%%%%%%%%%%%%%%%%%%%%%%%%%%%%

\documentclass[magistrale,ringraziamenti,linea,titoliSinistra,listaFigureFine,listaTabelleFine]{DarioBattistaGhilardi_Styles}


%% Completare con le informazioni relative alla tesi
%%
\TitoloTesi{Tecniche di analisi statica per la sicurezza di applicazioni web: problematiche ed implementazioni}
\Relatore{Prof. Marco Cremonini}
\Candidato{Dario Battista Ghilardi}
\Matricola{753708}
\AnnoAccademico{Anno Accademico 2010/2011}

%% introduzione di pacchetti aggiuntivi


%% introduzione di comandi ad hoc
%% qui e' possibile aggiungere altri providecommand!
%% 

%% \sphline puo' essere usato in alternativa a \hline per lo 
%% spazio tra due righe nella stessa tabella
%% 
\providecommand{\sphline}{
  \noalign{\vskip2pt}\\
 \hline
  \noalign{\vskip2pt}\\
}

%%%%%%%%%%%%%%%%%%%%%%%%%%%%%%%%%%%%%%%%%%%%%%%%%%%%%%%%%%%%%%%%%%%%%%%%%

%% sezione include
%% il comando \includeonly elenca i files con il sorgente latex da 
%% includere in fase di creazione del testo tra quelli elencati in 
%% seguito col comando \include

\includeonly{
	capitolo1,
	capitolo2,
	capitolo3,
	capitolo4,
	capitolo5,
	capitolo6,
	%appendice
}

\begin{document}


%%%%%%%%%%%%%%%%%%%%%%%%%%%%%%%%%%%%%%%%%%%%%%%%%%%%%%%%%%%%%%%%%%%%%%%%%

%% SEZIONE TESTO DELLA TESI

%% usare il comando part se si vuole separare la tesi in parti distinte
%% esempio: \part{Titolo Parte}

\chapter{Introduzione}
Solo 15 anni fa nessuno avrebbe mai pensato che Internet sarebbe diventata parte indispensabile per la vita quotidiana di ogni individuo. Da semplice meccanismo di condivisione delle informazioni mediante pagine statiche, la rete si è evoluta fino a diventare una realtà dinamica ed interattiva, in grado di interagire con l'utente.
Il merito di tale risultato è senza dubbio da attribuire alla quantità di applicazioni disponibili: da sistemi per gestire il proprio patrimonio economico da remoto, a sistemi per acquistare e vendere oggetti, a sistemi per condividere fotografie e per gestire la propria rete sociale.\\
Le applicazioni web hanno però introdotto una nuova importante problematica: la sicurezza dei dati sensibili degli utenti. Diverse tipologie di vulnerabilità possono portare alla perdita di tali dati, a modifiche nel funzionamento dell'applicazione ed alla compromissione del server sul quale essa risiede. Sebbene tale rischi siano ingenti, per molteplici cause quali il rispetto delle tempistiche, l'assenza di adeguate conoscenze di sicurezza e di programmazione, questi vengono spesso sottostimati dagli sviluppatori e la presenza di vulnerabilità non è rara. \\
Al fine di ridurre il rischio, le applicazioni web vengono spesso riviste manualmente da team di esperti in grado di identificarne le vulnerabilità. Spesso capita che i limiti imposti dal budget portino ad una riduzione nell'accuratezza dell'operazione di review, compito di per se lungo e con alto margine di errore.\\
Esistono però alcuni strumenti in grado di fornire soccorso a chi esegue la review del codice, incrementando la velocità del processo e riducendone di conseguenza i costi. Tali strumenti si occupano di eseguire un'analisi statica del codice sorgente dell'applicazione, ovvero verificano determinate proprietà senza che il codice venga eseguito.\\
In questo lavoro di tesi vengono analizzate le metodologie utilizzate per l'analisi statica, con vantaggi, svantaggi e con particolare approfondimento sull'implementazione di alcuni strumenti rivolti all'analisi di codice in linguaggio PHP.\\
Viene poi in seguito introdotto Vulture, un tool sperimentale per l'esecuzione di analisi statica di codice PHP. Vulture ha l'obiettivo di automatizzare il processo di scoperta di potenziali vulnerabilità nel codice sorgente al fine di fornire allo sviluppatore un report che semplifica l'operazione di review. Verranno discusse le decisioni architetturali prese durante la realizzazione del tool e le problematiche incontrate in fase di realizzazione. 


\chapter{Sicurezza di applicazioni web}

\begin{epigraphs}
\qitem{"If you think technology can solve your security problems, then you don't understand the problems and you don't understand the technology."}{---\textsc{ Bruce Schneier, 'Applied Cryptography' author}}
\end{epigraphs}

La diffusione globale di internet è un fenomeno in costante crescita, determinato dalle sempre più agevoli condizioni di accesso e coadiuvato dall'interesse per i servizi che la rete offre. \\
Le applicazioni web sono parte fondamentale di questo processo, la loro evoluzione nel corso degli anni è stata un fattore determinante per la crescita della rete. Servizi sempre più complessi hanno favorito l'interazione con gli utenti e la crescita di nuove opportunità di business. Ciò ha catturato l'interesse di individui con cattive intenzioni, per tale motivo è nata l'esigenza di meccanismi che potessero assicurare la sicurezza dei dati.

Un'applicazione web consiste in un software, posizionato su un server, al quale si accede tramite un'interfaccia web. L'accesso si basa sul protocollo HTTP ed è effettuato da vari clients. Ad ogni interazione tra il server ed il client viene stabilita una comunicazione, che consiste in una richiesta HTTP (solitamente GET e POST) con una serie di parametri che stabiliscono i valori della richiesta.\\
Il server riconosce i valori che vengono scambiati e di conseguenza costruisce delle risposte coerenti con le richieste, a seconda delle istruzioni definite nell'applicazione, a loro volta create tramite il linguaggio server-side utilizzato. Il codice sorgente dell'applicazione può fare uso dei parametri forniti per effettuare molteplici operazioni, da query sul database a lettura di file, fino a chiamate sul sistema operativo. Tipicamente il linguaggio server-side produce una risposta in formato HTML, che viene poi spedita al client e mostrata nel browser.\\
I parametri che vengono scambiati tra client e server però non sono sempre prevedibili, gli utenti possono inserire parametri che portano a comportamenti inattesi nell'applicazione. Tali comportamenti possono essere innocui oppure possono essere dannosi per l'applicazione stessa o per gli utenti che la utilizzano. Possono infatti svelare dati sensibili, compromettere l'applicazione o compromettere le altre applicazioni che lavorano sullo stesso server. 

Creare sistemi sicuri non è semplice e comporta la soluzione di numerosi e complessi problemi: dallo sviluppo di un'architettura sicura alla creazione di robusti sistemi crittografici fino alla definizione di policy di sicurezza. Nonostante l'esistenza di queste problematiche, grossa parte delle problematiche di sicurezza di un'applicazione sono dovute ad errata implementazione oppure alla negligenza dello sviluppatore.\\
Nel corso degli anni il problema della sicurezza dei dati ha assunto dimensioni rilevanti tanto che sono state proposte soluzioni per integrare la sicurezza nel processo di sviluppo software.

\section{Fondamenti di sicurezza}
La sicurezza nel software è basata sui principi di \emph{confidentiality}, \emph{integrity} ed \emph{availability}, solitamente definita con l'acronimo \emph{CIA}.\\
\begin{itemize}
\item Confidentiality: indica il divieto di diffusione di informazioni a soggetti o sistemi non autorizzati. E' condizione necessaria (ma non sufficiente) per garantire la privacy.
\item Integrity: indica la certezza che un dato non venga modificato in modo imprevisto da chi non ne possiede l'autorizzazione.
\item Availability: indica la disponibilità del dato per chi ne ha l'autorizzazione quando richiesto.
\end{itemize}
Negli ultimi anni tuttavia, con la motivazione di non riuscire a caratterizzare completamente il termine, è stata messa in discussione la definizione di sicurezza; sono stati proposti ulteriori termini che tengono conto anche di altri parametri. Nel 2002 Donn Parker\cite{parker} propose un'alternativa composta da sei termini, aggiungendo ai tre classici principi le nozioni di possession, authenticity e utility (la sestupla verrà in seguito nominata \emph{Parkerian Hexad}).
Possession indica la proprietà dei diritti di controllo dei dati, authenticity indica la capacità di accertare la validità del dato, utility indica la capacità di un dato di essere utile per un determinato scopo.

\section{Vulnerabilità nelle applicazioni web}
La presenza di vulnerabilità nel software è dovuta a vari motivi: errate decisioni architetturali ed implementative, mancata conoscenza da parte dello sviluppatore delle problematiche di security, negligenza. Queste ultime due motivazioni sono ancora più veritiere nel mondo degli applicativi web: linguaggi come PHP non hanno una curva di apprendimento ripida e consentono anche ai non professionisti di realizzare applicazioni web.\\
Molti sviluppatori non conoscono o non si rendono conto delle problematiche di security a cui vanno incontro se vengono inseriti dati malevoli nelle loro applicazioni. E' un problema principalmente di educazione, i vari libri di programmazione difficilmente si soffermano sull'importanza di scrivere codice sicuro. Allo stesso modo il lavoro di sviluppatore non sempre richiede certificazioni in ambito sicurezza per essere praticato.\\
Un'altra motivazione che esclude la sicurezza dal processo di sviluppo software è costituita da restrizioni economiche, le quali possono incidere sulle tempistiche e quindi diminuire il tempo da dedicare al testing ed al controllo del codice. Solitamente infatti raggiungere una release stabile del progetto è la massima priorità, mentre la sicurezza non lo è.

Il controllo qualità nei progetti software è focalizzato solitamente sull'adesione ai requisiti imposti in fase di progettazione, non sulle implicazioni che può avere una errata implementazione delle specifiche. Solitamente però la presenza di vulnerabilità non indica una violazione delle specifiche imposte dai requisiti.\\
Tecnicamente, al fine di rendere un software sicuro, tutte le parti di quel software devono essere sicure, non solo le parti sensibili dal punto di vista della sicurezza come l'autenticazione o la gestione dei pagamenti. E' proprio in questo codice che statisticamente si concentrano le vulnerabilità, quelle in cui la sicurezza non è un requisito. \\
Un esempio di tale eventualità si ha con la procedura di acquisto prodotti su un e-commerce: non è necessario che solo la parte di acquisto tramite carta di credito sia sicura, un attaccante può sfruttare una vulnerabilità in qualunque punto dell'applicazione per accedere ai dati delle carte di credito degli utenti.

Le vulnerabilità più comuni all'interno di un'applicazione web sono chiamate \emph{taint-style vulnerabilities}. Il nome deriva dal fatto che un dato (\emph{tainted data}) entra nel flusso del programma attraverso una sorgente non fidata e viene passato ad una porzione del programma vulnerabile (detta \emph{sensitive sink}) senza essere prima sanitizzato da una opportuna routine (\emph{sanitization routine}). La mancata sanitizzazione del dato comporta la presenza di una vulnerabilità di tipo taint-style.\\
Il linguaggio di scripting Perl possiede un \emph{taint-mode}, ovvero una modalità che considera ogni input dell'utente come tainted. Con taint-mode attivo, solo input da provenienti da sinks che vengono validati attraverso espressioni regolari vengono considerati validi.\\
Questa tecnica ha un evidente grosso punto debole: le espressioni regolari sono molto complesse ed è facile per uno sviluppatore commettere un errore che possa invalidare la sanitization routine (Jovanovic et. al.\cite{pixy}). Per tale motivo è sconsigliato l'uso di sanitization routine basate su espressioni regolari definite in modo personalizzato ma si consiglia di usare i costrutti forniti dal linguaggio. L'assunzione che indica come sicuro ogni valore che viene passato attraverso un'espressione regolare è oltremodo problematica, fornisce un falso senso di sicurezza, per tale motivo non è sensato riprodurre il taint-mode di Perl su altri linguaggi come misura protettiva.

Le vulnerabilità di tipo taint-style, tra cui injections, cross site scripting e cross site request forgery, sono le più comuni ma non sono le uniche ad affliggere le applicazioni web.
OWASP (Open Web Application Security Project)\footnote{http://www.owasp.org}, un gruppo composto da volontari che produce tools, standard e documentazione open-source gratuita inerente la web security, ha categorizzato le varie vulnerabilità nella sua Top Ten\cite{topten}, un progetto rilasciato con cadenza triennale che raccoglie 10 vulnerabilità tipiche di applicazioni web. 
Gli obiettivi di OWASP sono i seguenti:
\begin{itemize}
\item Diffondere la cultura dello sviluppo di applicativi web sicuri.
\item Contribuire alla sensibilizzazione sia dei professionisti che delle aziende verso le problematiche di web security, attraverso la circolazione di idee, articoli, best-practices e tools.
\item Promuovere l’uso di metodologie e tecnologie che consentano di migliorare il livello di sicurezza delle applicazioni web.
\end{itemize}
OWASP Top Ten è una classificazione accettata a livello globale basata ed è basata sul rischio, che fornisce anche le contromisure per mitigare l'eventuale problematica. Di seguito si riportano le vulnerabilità citate dalla OWASP Top Ten 2010, utili in seguito nella trattazione, corredate da un esempio di come è possibile riscontrarle nell'applicazione web (sebbene l'esempio sia didattico e meno complicato rispetto alla vulnerabilità riscontrata negli applicativi professionali).

\subsection{Injection}
Questa categoria di vulnerabilità raccoglie tutte le casistiche in cui dati non fidati vengono inviati ad un interprete come parte di un comando o di una query. Tali dati possono essere eseguiti dall'interprete e possono condurre all'esecuzione di comandi non voluti oppure all'accesso a dati non autorizzati. Fanno parte di questa categoria le vulnerabilità di tipo SQL injection, LDAP injection e OS injection.\\
E' molto comune trovare questa tipologia di vulnerabilità in codice legacy, ovvero non più supportato, ed è una vulnerabilità che ha un severo impatto sull'applicazione poichè può portare alla corruzione del sistema, ad un \emph{denial of service} oppure alla perdita di dati.\\
Di seguito si riporta un esempio di tale vulnerabilità:\\

\begin{lstlisting}[language=SQL]
$query = "SELECT * FROM accounts WHERE custID ='" . $_GET["id"] ."'";
mysql_query($query);
\end{lstlisting}

L'applicazione esegue la query sul database MySql sottostante, utilizzando come parametro un valore ottenuto direttamente dall'URL. Tale query espone però l'applicazione ad un possibile attacco di tipo SQL Injection. Infatti inserendo nell'URL una stringa come la seguente\\

\begin{lstlisting}[language=SQL]
http://example.com/app/accountView?id=' or '1'='1
\end{lstlisting}

la query viene interpretata in modo diverso, ritornando tutti i record di quella tabella dal database. Nel caso peggiore un attaccante può utilizzare questa vulnerabilità per eseguire query che alterano i dati nel database, riuscendo ad ottenere il completo controllo dell'applicazione. \\
Per evitare di incorrere in questa tipologia di vulnerabilità è opportuno utilizzare un API per il dialogo con il database, che si occupa di filtrare i parametri in ingresso alle query. Una soluzione alternativa può essere invece quella di effettuare l'escape dei caratteri speciali usando specifiche sintassi per ogni interprete.

\subsection{Cross site scripting}
Vulnerabilità di tipo Cross site scripting (denominate spesso con l'acronimo XSS, da non confondere con CSS di cascading style sheets) si verificano quando un'applicazione riceve dati di input non fidati e li invia ad un browser senza un'appropriata procedura di controllo e validazione. \\
XSS consente ad un attaccante di eseguire scripts sul browser della vittima, i quali possono effettuare l'hijacking della sessione utente, possono recuperare cookie di sessione, possono redirezionare l'utente su siti web malevoli o possono effettuare defacing del sito web. \\
Un esempio di cross site scripting può essere il seguente:\\

\begin{lstlisting}[language=PHP]
$page += "<input name='creditcard' type="text" value='" + $_GET["CC"] + "'>";
\end{lstlisting}

Supponendo di avere una pagina che mostra a video il numero di carta di credito di un individuo, ottenuto attraverso i parametri in input dall'URL, l'attaccante potrà semplicemente costruire un URL con un valore del parametro \emph{CC} modificato e fare in modo che l'utente visiti tale URL per effettuare l'hijacking della sessione utente, come nell'esempio seguente:\\

\begin{lstlisting}[language=PHP]
'><script>document.location= 'http://www.attacker.com/cgi-bin/cookie.cgi?foo='+document.cookie</script>'.
\end{lstlisting}

Esistono tre diverse tipologie di cross site scripting:
\begin{itemize}
\item Stored: Il codice viene iniettato nel server in modo permanente, ad esempio in un database. La vittima riceve quindi lo script malevolo ad ogni visita della pagina che mostra a video quel valore.
\item Reflected: Il codice malevolo non viene iniettato nel server ma viene inviato alla vittima attraverso mezzi alternativi, come un email contenente un link. Quando l'utente clicca su tale link il codice viene eseguito.
\item DOM based: Un payload malevolo viene eseguito come risultato della modifica del DOM\footnote{Document Object Model} del browser dell'utente. La risposta HTTP in questo caso non cambia ma il codice contenuto nella pagina viene eseguito in modo diverso a causa delle modifiche al DOM.\\
E' diverso da stored e reflected XSS poichè in questo caso il payload malevolo non è nella pagina di risposta del server ad una richiesta.
\end{itemize}

Vulnerabilità di tipo XSS hanno impatto significativo sull'utente, meno significativo sull'applicazione (ad eccezione del caso stored, in cui l'applicazione è direttamente coinvolta).\\
Prevenire vulnerabilità di tipo XSS comporta la separazione tra dati non fidati ed il contenuto che viene inviato al browser. E' quindi necessario effettuare l'escape dei contenuti in input basati su codice HTML o Javascript, a seconda del contesto in cui tali dati verranno poi utilizzati.

\subsection{Broken authentication and session management}
Funzionalità come l'autenticazione e la gestione delle sessioni utente sono spesso implementate non correttamente, consentendo all'attaccante di ottenere passwords, chiavi, tokens di sessione o di impersonificare altri utenti.\\
Il classico esempio di questa vulnerabilità si verifica quando il timeout della sessione utente non è settato correttamente. Supponendo che l'utente sia loggato nell'applicazione attraverso un browser su un computer e al termine dell'uso si dimentichi di cliccare su logout, un attaccante potrebbe collegarsi al sito attraverso lo stesso computer e ritrovarsi già loggato nell'applicazione, con il profilo del vecchio utente. \\
Questa tipologia di vulnerabilità ha radici architetturali oltre che implementative, per tale motivo le contromisure consistono nel seguire raccomandazioni e specifiche per la gestione delle sessioni e dell'autenticazione utente.

\subsection{Insecure direct object references}
Una direct object reference si verifica quando uno sviluppatore espone un collegamento ad un oggetto interno, come un file, una directory, una chiave per accedere al database, ecc. Senza le opportune protezioni gli attaccanti possono manipolare tali collegamenti per accedere a dati in modo non autorizzato.\\
Un esempio di tale vulnerabilità può verificarsi nelle applicazioni di home banking, dove il numero di conto corrisponde alla chiave primaria nel database. Anche se gli sviluppatori hanno utilizzato query SQL che evitano injections, se non esistono controlli aggiuntivi per verificare che l'utente sia il proprietario dell'account e sia autorizzato a visualizzare determinati contenuti, un attaccante potrebbe sfruttare il numero di conto per visualizzare dati provenienti da conti altrui.\\
Le contromisure consistono nella verifica continua che l'accesso ai dati esposti avvenga effettivamente dagli utenti autorizzati e nell'uso di referenze ad oggetti interni indirette, quali ad esempio indici non collegati a dati sensibili. 

\subsection{Cross site request forgery}
Un attacco di tipo cross site request forgery (CSRF) forza una vittima loggata nell'applicazione web ad eseguire richieste HTTP costruite dall'attaccante per uno scopo ben preciso (tramite tags di tipo immagine, XSS o altre tecniche), generalmente ad insaputa della vittima. L'applicazione a questo punto reagisce interpretando la richiesta come legittima, e la richiesta può accedere ad ogni dato a cui la vittima può accedere.\\
Si riporta un esempio di tale problematica:\\

\begin{lstlisting}[language=PHP]
http://example.com/app/transferFunds?amount=1500&destinationAccount=4673243243
\end{lstlisting}

L'applicazione di home banking esegue un trasferimento fondi da un account ad un altro mediante l'uso dei parametri riportati. In questo caso non vi è nulla di segreto nella richiesta. Se l'attaccante fosse in grado di lanciare sul computer della vittima loggata al sito una richiesta di questo tipo, con il proprio numero di conto nel campo destinationAccount, potrebbe trasferire la cifra riportata a se stesso. Per fare ciò, potrebbe forgiare un tag di tipo immagine apposito e fare in modo che la vittima lo visualizzi da loggata.\\
L'esempio sottostante mostra un tag immagine adatto allo scopo:\\

\begin{lstlisting}[language=PHP]
<img src="http://example.com/app/transferFunds?amount=1500&destinationAccount=attackersAcct#" width="0" height="0" />
\end{lstlisting}

Prevenire CSRF comporta la creazione di un token non prevedibile nel corpo o nell'URL di ogni richiesta HTTP. Tale token dovrebbe essere univoco per sessione utente, ma è ancora meglio se è univoco per ogni richiesta. Con la presenza di tale valore non è più possibile per l'attaccante la creazione di una richiesta valida da fare eseguire di nascosto alla vittima.

\subsection{Altre vulnerabilità}
Le altre vulnerabilità citate in OWASP Top Ten ma non riportate nella sezioni precedenti sono le seguenti:

\begin{itemize}
\item \emph{Security misconfiguration}: Le applicazioni web si basano su uno stack, ovvero un insieme di programmi che lavorano a diversi livelli. Molti di essi devono essere configurati correttamente, poiché non tutti vengono forniti di default con un setup sicuro. Inoltre devono essere mantenuti ed aggiornati. Questa tipologia di vulnerabilità include tutti i casi in cui tale software non viene configurato o mantenuto correttamente.
\item \emph{Insecure cryptographic storage}: Molte applicazioni web non proteggono attraverso l'uso di cifratura i dati sensibili, come numeri di carte di credito o credenziali di autenticazione. Gli attaccanti possono quindi rubare o modificare tali dati per eseguire furti di identità, frodi ed altri crimini.
\item \emph{Failure to restrict URL access}: E' comune per un'applicazione web controllare i permessi di accesso all'URL prima di visualizzare contenuti protetti. Tuttavia le applicazioni necessitano che questi controlli vengano eseguiti ogni volta che queste pagine vengono visualizzate o gli attaccanti saranno in grado di costruire URL appositi per accedere a tali risorse senza i permessi.
\item \emph{Insufficient transport layer protection}: Le applicazioni spesso falliscono nel proteggere la confidenzialità e l'integrità del traffico di rete. Questo poiché talvolta utilizzano algoritmi non corretti, certificati non validi o scaduti, non implementano SSL oppure falliscono nell'implementare le best practice per questa problematica.
\item \emph{Unvalidated redirects and forwards}: Può capitare che un'applicazione redirezioni l'utente verso altre pagine o siti web, usando dati non sicuri per determinare le pagine di destinazione. Senza l'appropriata validazione un attaccante può redirezionare la vittima verso phishing verso siti contenenti malware.
\end{itemize}

\subsection{Http Parameter Pollution}
Http Parameter Pollution (HPP) è una vulnerabilità presentata per la prima volta nel 2009 da Stefano di Paola e Luca Carrettoni\cite{dipaola} ad OWASP AppSec Europe 2009. Sebbene troppo recente per essere inclusa nella OWASP Top Ten, questa vulnerabilità è la motivazione che ha portato alla nascita di Vulture, il tool che verrà presentato successivamente in questa tesi.\\
HPP, descritta in modo esaustivo da Balduzzi\cite{hppbalduzzi}, consiste nell'inserimento nell'input di un'applicazione web una particolare query string a parametri encodati, la quale, se l'applicazione non esegue una sanitizzazione dei valori in input, può consentire ad un utente malevolo di comprometterne la logica.

La RFC 3986\cite{rfc3986} specifica che la query string è la parte dell'URI compresa tra il carattere "?" e la fine dell'URI stesso. Normalmente la query string è composta da una serie di coppie chiave=valore che identificano i valori in input, separate da un carattere "\&" oppure ";". Al fine di evitare incomprensioni con i separatori, ogni carattere speciale viene encodato in forma esadecimale.	\\
A seconda della tecnologia in cui l'applicazione è stata sviluppata, la query string viene però interpretata in modo differente: se esistono coppie chiave=valore con la stessa chiave, l'applicazione può prendere in input il primo dei due valori, l'ultimo, oppure una combinazione di entrambi.\\
Nella tabella \ref{hpp} si mostra in dettaglio il funzionamento dell'applicazione a seconda della tecnologia usata.

\begin{table}[!h]
\center
\begin{tabular}{|c|c|c|}
\hline
Tecnologia e server & Metodo & Precedenza \\
\hline
ASP/IIS & Request.QueryString("par") & Tutti (separati da virgole)\\
PHP/Apache & \$\_GET["par"] & Ultimo valore \\
JSP/Tomcat & Request.getParameter("par") & Primo valore \\
Perl(CGI)/Apache & Param("par") & Primo valore \\
Python/Apache & getValue("par") & Tutti (in una lista)\\
\hline
\end{tabular}
\caption{Precedenza in caso di parametri con lo stesso nome}
\label{hpp}
\end{table}


La presenza di più valori per la stessa chiave della query string può portare a comportamenti indesiderati dell'applicazione, è quindi fonte di attacchi da parte di utenti malintenzionati.\\
Uno scenario tipico consiste nel far si che la vittima dell'attacco venga redirezionata verso un URL che sfrutta una vulnerabilità di tipo HPP. Supponendo che l'applicazione in questione sia un sistema scritto in JSP per individuare il giorno migliore in cui effettuare un meeting, che riceve in input un singolo parametro chiamato \emph{meeting\_id}, al fine di identificare univocamente il meeting stesso, l'attaccante potrebbe effettuare HPP sulla query string con lo scopo di modificare i valori dei link che segnalano il giorno di preferenza dell'utente.
Nel normale funzionamento dell'applicazione i link generati sono i seguenti:\\

\begin{lstlisting}[language=HTML]
http://example.com/choice.jsp?meeting_id=1234

Link 1: <a href="http://example.com/choice.jsp?meeting_id=1234&day=20111201">1 dicembre 2011</a>
Link 1: <a href="http://example.com/choice.jsp?meeting_id=1234&day=20111202">2 dicembre 2011</a>
\end{lstlisting}

L'attaccante, aggiungendo alla query string il valore encodato \emph{day=20111202}, rende la scelta dell'utente indifferente grazie all'attacco di tipo HPP:\\

\begin{lstlisting}[language=HTML]
http://example.com/choice.jsp?meeting_id=1234%26day%3D20111202

Link 1: <a href="http://example.com/choice.jsp?meeting_id=1234\%26day\%3D20111202&day=20111201">1 dicembre 2011</a>
Link 1: <a href="http://example.com/choice.jsp?meeting_id=1234\%26day\%3D20111202&day=20111202">2 dicembre 2011</a>
\end{lstlisting}

Nel secondo caso, visto che il funzionamento di JSP con i doppi parametri in query string consiste nel scegliere sempre il primo valore tra quelli con la stessa chiave, qualunque link l'utente clicchi la sua scelta sarà sempre salvata come 1 dicembre 2011.\\
Esistono alcune varianti di HPP che si rifanno però sempre all'injection di valori multipli nella query string ed alle precedenze nell'uso di tali valori da parte dell'applicazione. E' quindi compito dello sviluppatore controllare la reazione dell'applicazione all'invio di parametri multipli.

\section{Security development life cycle}
Tutte le realtà aziendali che si occupano di software implementano un modello che definisce le fasi entro cui lo sviluppo si verifica. Tali fasi sono definite nel software development life cycle (SDLC).\\
Esistono diversi modelli di software development life cycle:
\begin{itemize}
\item A cascata
\item A spirale
\item Iterativo ed incrementale
\item Agile
\item Code and fix
\end{itemize}
Tutti questi modelli però non contemplano la fase di analisi della sicurezza dell'applicazione, bensì relegano tale fase al termine del processo di sviluppo. Ciò è inefficiente poiché influisce sui costi (una modifica a prodotto completato è sicuramente più costosa di una modifica eseguita durante lo sviluppo) e comporta una perdita di controllo (modificare il prodotto in uno stage così avanzato del processo comporta dover eseguire nuovamente tutti i test).
L'immagine \ref{pep} mostra le problematiche del modello patch-and-penetrate.

\begin{figure}[!h]
\centering
\includegraphics[width=15cm]{patchpenetrate.jpg}
\caption{Modello patch-and-penetrate}\label{pep}
\end{figure}


Per tali motivi è opportuno che l'analisi di sicurezza venga implementata nello sviluppo, introducendo un diverso software development life cycle.

\begin{figure}[!h]
\centering
\includegraphics[width=15cm]{sdlc.jpg}
\caption{Software development life cycle}\label{sdlc}
\end{figure}

L'immagine \ref{sdlc} mostra come la sicurezza non sia una fase all'interno del processo di sviluppo, bensì un processo, da attuare costantemente.\\
E' fondamentale identificare il prima possibile un problema di sicurezza, poiché ciò comporta costi minori di fixing e riduce in modo drastico la finestra di esposizione, presente invece nel modello patch-and-penetrate, dominante fino a pochi anni fa. \\
Introdurre l'analisi di sicurezza nel SDLC non è però semplice, occorre una maggiore conoscenza delle problematiche e qualche sistema in grado di automatizzare la ricerca di errori comuni, che abbia come requisito la rapidità di scansione e che possa fornire feedbacks immediati agli sviluppatori.\\
Per questo compito, oltre alla sempre necessaria formazione degli sviluppatori, l'analisi statica può rivelarsi un aiuto efficace.

\chapter{Sicurezza di applicazioni web}

\begin{epigraphs}
\qitem{"If you think technology can solve your security problems, then you don't understand the problems and you don't understand the technology."}{---\textsc{ Bruce Schneier, 'Applied Cryptography' author}}
\end{epigraphs}

La diffusione globale di internet è un fenomeno in costante crescita, determinato dalle sempre più agevoli condizioni di accesso e coadiuvato dall'interesse per i servizi che la rete offre. \\
Le applicazioni web sono parte fondamentale di questo processo, la loro evoluzione nel corso degli anni è stata un fattore determinante per la crescita della rete. Servizi sempre più complessi hanno favorito l'interazione con gli utenti e la crescita di nuove opportunità di business. Ciò ha catturato l'interesse di individui malintenzionati, per tale motivo è nata l'esigenza di meccanismi che potessero garantire la sicurezza dei dati.

Creare sistemi sicuri comporta la soluzione di numerosi e complessi problemi: dallo sviluppo di un'architettura sicura alla creazione di robusti sistemi crittografici fino alla definizione di policy di sicurezza. Nonostante l'esistenza di queste problematiche, grossa parte degli attacchi alla security di un'applicazione sono rivolti all'errata implementazione del software oppure a vulnerabilità create dalla negligenza dello sviluppatore.
Nel corso degli anni il problema della sicurezza dei dati ha assunto dimensioni rilevanti tanto che sono state proposte soluzioni per integrare la security nel processo di sviluppo software.

\section{Fondamenti di sicurezza}
La sicurezza nel software è basata sui principi di \emph{confidentiality}, \emph{integrity} ed \emph{availability}, solitamente caratterizzata dell'acronimo \emph{CIA}.\\
\begin{itemize}
\item Confidentiality: indica la misura che vieta la diffusione di informazioni a soggetti o sistemi non autorizzati. E' condizione necessaria (ma non sufficiente) per garantire la privacy.
\item Integrity: indica la certezza che un dato non venga modificato in modo imprevisto da chi non ne possiede l'autorizzazione.
\item Availability: indica la disponibilità del dato per chi ne ha l'autorizzazione quando richiesto.
\end{itemize}
Negli ultimi anni tuttavia è stata messa in discussione la definizione di sicurezza attraverso questi tre termini, con la proposta di termini aggiuntivi. Ad esempio nel 2002 Donn Parker\cite{CITAZIONE} propose un'alternativa composta da sei termini, aggiungendo ai tre classici principi le nozioni di possession, authenticity e utility.
Possession indica la proprietà dei diritti di controllo dei dati, authenticity indica la capacità di accertare la validità del dato, utility indica la capacità di un dato di essere utile per un determinato scopo.

\section{Vulnerabilità nelle applicazioni web}
Le vulnerabilità sono presenti nel software per vari motivi: per errate decisioni architetturali ed implementative, per mancata conoscenza da parte dello sviluppatore delle problematiche di security e per negligenza. Queste ultime due motivazioni sono ancora più veritiere nel mondo degli applicativi web: linguaggi come PHP non hanno una curva di apprendimento ripida e consentono a chiunque di realizzare applicazioni web.\\
Molti sviluppatori non conoscono o non si rendono conto delle problematiche di security a cui vanno incontro se vengono inseriti dati malevoli nelle loro applicazioni. E' un problema principalmente di educazione, i vari libri di programmazione difficilmente si soffermano sull'importanza di scrivere codice sicuro. Allo stesso modo il lavoro di sviluppatore non sempre richiede determinate certificazioni per essere praticato.\\
Un'altra motivazione che esclude la sicurezza dal processo di sviluppo software è costituita dalle condizioni economiche, le quali possono incidere sulle tempistiche e quindi restringere il tempo da dedicare al testing ed al controllo del codice. Solitamente infatti raggiungere una release stabile del progetto è la massima priorità, mentre la sicurezza non lo è.\\

Il controllo qualità nei progetti software è altamente focalizzato sull'adesione ai requisiti imposti in fase di progettazione, molto meno sulle implicazioni che può avere una errata implementazione delle specifiche. Solitamente anche in presenza di vulnerabilità non si violano le specifiche imposte dai requisiti.\\
Tecnicamente, al fine di rendere un software sicuro, tutte le parti di quel software devono essere sicure, non solo le parti sensibili dal punto di vista della sicurezza. E' proprio in questo codice che statisticamente si concentrano le vulnerabilità, quelle in cui la sicurezza non è un requisito. \\
Un esempio di tale situazione è la procedura di acquisto prodotti su un e-commerce: non è necessario che solo la parte di acquisto tramite carta di credito sia sicura, un attaccante può sfruttare una vulnerabilità in qualunque punto dell'applicazione per accedere ai dati delle carte di credito degli utenti.

OWASP (Open Web Application Security Project) è un gruppo composto da volontari che produce tools, standard e documentazione open-source gratuita inerente la web security. 
Gli obiettivi di OWASP sono i seguenti:
\begin{itemize}
\item Diffondere la cultura dello sviluppo di applicativi web sicuri.
\item Contribuire alla sensibilizzazione sia dei professionisti che delle aziende verso le problematiche di Web Security, attraverso la circolazione di idee, articoli, best- practice e tool.
\item Promuovere l’uso di metodologie e tecnologie che consentano di migliorare il livello di sicurezza delle applicazioni web.
\end{itemize}
OWASP Top Ten è un progetto, rilasciato da OWASP con cadenza triennale, che raccoglie 10 tipologie di vulnerabilità tipiche delle applicazioni web. E' una classificazione accettata a livello globale basata sul rischio, che fornisce anche le contromisure per mitigare l'eventuale problematica. Di seguito si riportano le prime cinque vulnerabilità citate dalla OWASP Top Ten 2010, utili in seguito nella trattazione.

\subsection{Injection}
Questa categoria di vulnerabilità raccoglie tutte le casistiche in cui dati non fidati vengono inviati ad un interprete come parte di un comando o di una query. Tali dati possono essere eseguiti dall'interprete e possono condurre all'esecuzione di comandi non voluti oppure all'accesso a dati non autorizzati. Fanno parte di questa categoria SQL injection, LDAP injection e OS injection.\\
E' molto comune trovare questa tipologia di vulnerabilità in codice legacy, ovvero non più supportato dal produttore, ed è una vulnerabilità che ha un severo impatto sull'applicazione poichè può portare alla corruzione del sistema, ad un \emph{denial of service} oppure alla perdita di dati.\\
Un esempio di tale vulnerabilità può essere il seguente:
\begin{lstlisting}[language=SQL]
$query = "SELECT * FROM accounts WHERE custID ='" . $_GET["id"] ."'";
mysql_query($query);
\end{lstlisting}

L'applicazione esegue la query sul database MySql sottostante, utilizzando come parametro un valore preso direttamente dall'URL. Tale query espone però l'applicazione ad un possibile attacco di tipo SQL Injection. Infatti inserendo nell'URL una stringa come la seguente

\begin{lstlisting}[language=SQL]
http://example.com/app/accountView?id=' or '1'='1
\end{lstlisting}

la query viene interpretata in modo diverso, ritornando tutti i record di quella tabella dal database. Nel caso peggiore un attaccante può utilizzare questa vulnerabilità per eseguire query che alterano i dati nel database, riuscendo ad ottenere il completo controllo. \\
Per evitare di incorrere in questa tipologia di vulnerabilità è opportuno utilizzare un API per il dialogo con il database, che si occupa di filtrare i parametri in ingresso alle query. Una soluzione alternativa può essere quella di effettuare l'escape dei caratteri speciali usando specifiche sintassi per ogni interprete.

\subsection{Cross site scripting}
Vulnerabilità di tipo Cross site scripting (denominate spesso con l'acronimo XSS, da non confondere con CSS di cascading style sheets) si verificano quando un'applicazione riceve dati di input non fidati e li invia ad un browser senza un'appropriata validazione o escaping. \\
XSS consente ad un attaccante di eseguire scripts sul browser della vittima, i quali possono effettuare l'hijacking della sessione utente, possono recuperare cookie di sessione, possono redirezionare l'utente su siti web malevoli o possono effettuare defacing del sito web. \\
Un esempio di cross site scripting può essere il seguente: 

\begin{lstlisting}[language=PHP]
$page += "<input name='creditcard' type="text" value='" + $_GET["CC"] + "'>";
\end{lstlisting}

Supponendo di avere una pagina che mostra a video il numero di carta di credito di un individuo, ottenuto attraverso i parametri in input dall'URL, l'attaccante potrà semplicemente costruire un URL con un valore del parametro CC modificato e fare in modo che l'utente visiti tale URL per effettuare l'hijacking della sessione utente, come nell'esempio seguente:

\begin{lstlisting}[language=PHP]
'><script>document.location= 'http://www.attacker.com/cgi-bin/cookie.cgi?foo='+document.cookie</script>'.
\end{lstlisting}

Esistono tre tipologie di cross site scripting:
\begin{itemize}
\item Stored: Il codice viene iniettato nel server in modo permanente, in un database, in un forum, in un commento, ecc. La vittima ottiene lo script malevolo ad ogni visita della pagina.
\item Reflected: Il codice malevolo non viene iniettato nel server ma viene inviato alla vittima attraverso mezzi alternativi, come un email contenente un link. Quando l'utente viene convinto a cliccare su tale link il codice viene eseguito.
\item DOM based: Un payload malevolo viene eseguito come risultato della modifica del DOM\footnote{Document Object Model} del browser dell'utente. La risposta HTTP in questo caso non cambia ma il codice contenuto nella pagina viene eseguito in modo diverso a causa delle modifiche effettuate al DOM.\\
E' diverso da stored e reflected XSS poichè in questo caso il payload melevolo non è nella pagina di risposta del server ad una richiesta.
\end{itemize}

Vulnerabilità di tipo XSS hanno impatto significativo sull'utente, meno significativo sull'applicazione (ad eccezione del caso stored, in cui l'applicazione è direttamente coinvolta).\\
Prevenire vulnerabilità di tipo XSS comporta la separazione tra dati non fidati ed il contenuto attivo del browser. E' quindi necessario effettuare l'escape dei contenuti in input basati su codice HTML, a seconda del contesto in cui tali dati verranno poi utilizzati.

\subsection{Broken authentication and session management}
Funzionalità come l'autenticazione e la gestione delle sessioni utente sono spesso implementate non correttamente, consentendo all'attaccante di ottenere passwords, chiavi, tokens di sessione o di impersonificare altri utenti.\\
Il classico esempio di questa vulnerabilità si verifica quando il timeout della sessione utente non è settato correttamente. Supponendo che l'utente sia loggato nell'applicazione attraverso un browser su un computer e al termine dell'uso si dimentichi di cliccare su logout. Un attaccante potrebbe collegarsi al sito attraverso lo stesso computer e ritrovarsi già loggato nell'applicazione, con il profilo del vecchio utente. \\
Questa tipologia di vulnerabilità ha radici architetturali oltre che implementative, per tale motivo le contromisure consistono nel seguire raccomandazioni e specifiche per il management delle sessioni e dell'autenticazione utente.

\subsection{Insecure direct object references}
Una direct object reference si verifica quando uno sviluppatore espone un collegamento ad un oggetto interno, come un file, una directory, ad una chiave per accedere al database, ecc. Senza le opportune protezioni gli attaccanti possono manipolare tali collegamenti per accedere a dati in modo non autorizzato.\\
Un esempio di tale vulnerabilità può verificarsi nelle applicazioni di home banking, dove il numero di conto corrisponde alla chiave primaria nel database. Anche se gli sviluppatori hanno utilizzato query SQL che evitano injections, se non esistono controlli aggiuntivi per verificare che l'utente sia il proprietario dell'account e sia autorizzato a visualizzare determinati contenuti, un attaccante potrebbe sfruttare il numero di conto per visualizzare dati provenienti da conti altrui.\\
Le contromisure consistono nella verifica continua che l'accesso ai dati esposti avvenga effettivamente dagli utenti autorizzati e nell'uso di referenze ad oggetti interni indirette, quali ad esempio indici non collegati a dati sensibili. 

\subsection{Cross site request forgery}
Un attacco di tipo cross site request forgery (CSRF) forza una vittima loggata nell'applicazione web ad eseguire richieste HTTP costruite dall'attaccante per uno scopo ben preciso (tramite tags di tipo immagine, XSS o altre tecniche), generalmente ad insaputa della vittima. L'applicazione a questo punto reagisce interpretando la richiesta come legittima, e la richiesta può accedere ad ogni dato a cui la vittima può accedere.\\
Si riporta un esempio di tale problematica:

\begin{lstlisting}[language=PHP]
http://example.com/app/transferFunds?amount=1500&destinationAccount=4673243243
\end{lstlisting}

L'applicazione di home banking esegue un trasferimento fondi da un account ad un altro mediante l'uso dei parametri riportati. In questo caso non vi è nulla di segreto nella richiesta. Se l'attaccante fosse in grado di lanciare sul computer della vittima loggata al sito una richiesta di questo tipo, con il proprio numero di conto nel campo destinationAccount, potrebbe trasferire la cifra riportata a se stesso. Per fare ciò, potrebbe forgiare un tag di tipo img apposito e fare in modo che la vittima lo visualizzi da loggata.\\
L'esempio sottostante mostra un tag img adatto allo scopo:

\begin{lstlisting}[language=PHP]
<img src="http://example.com/app/transferFunds?amount=1500&destinationAccount=attackersAcct#" width="0" height="0" />
\end{lstlisting}

Prevenire CSRF comporta la creazione di un token non prevedibile nel corpo o nell'URL di ogni richiesta HTTP. Tale token dovrebbe essere univoco per sessione utente, ma è ancora meglio se è univoco per ogni richiesta. Con la presenza di tale valore non è più possibile per l'attaccante la creazione di una richiesta valida da fare eseguire di nascosto alla vittima.

\subsection{Altre vulnerabilità}
Le altre vulnerabilità citate in OWASP Top Ten ma non riportate nella sezioni precedenti sono le seguenti:

\begin{itemize}
\item Security misconfiguration: 
\item Insecure cryptographic storage: 
\item Failure to restrict URL access: 
\item Insufficient transport layer protection: 
\item Unvalidated redirects and forwards: 
\end{itemize} 

\section{Security development cycle}


\chapter{Analisi Statica}

\begin{epigraphs}
\qitem{"As soon as we started  
programming, we found to  
our surprise that it wasn’t as  
easy to get programs right as  
we had thought.  Debugging  
had to be discovered. I can 
remember the exact instant  
when I realized that a large  
part of my life from then on  
was going to be spent in finding 
mistakes in my own programs."}{---\textsc{ Maurice Wilkes, Inventore di EDSAC, 1949}}
\end{epigraphs}

Tutti i progetti software condividono una caratteristica fondamentale: possiedono un codice sorgente che ne definisce il funzionamento. Tale codice sorgente è costituito da una serie di istruzioni scritte in un linguaggio di programmazione che vengono interpretate da un compilatore e successivamente eseguite. Il codice sorgente di un software risiede tipicamente su uno o più files di testo.\\
Il codice sorgente non è esente da errori bensì ha l'intrinseca proprietà di possedere difetti. Sin dagli albori della programmazione software gli sviluppatori hanno avuto a che fare con tali difetti, individuando un rapporto di proporzionalità diretta tra il numero di questi ultimi ed il numero di righe di codice scritte per un determinato software. L'aumentare della complessità dei programmi e la necessità di affidabilità hanno reso il controllo dei difetti fondamentale nell'industria del software, tanto che è opportuno che prima di un rilascio determinati standard di qualità siano rispettati. \\
Al fine di ridurre il quantitativo di difetti nel software e di aderire agli standard i programmatori hanno pensato di sviluppare altro software in grado di analizzare il codice sorgente di un programma durante la fase di sviluppo.\\
Tale analisi è detta \emph{analisi statica} ed è una tecnica che consiste nell'ispezionare automaticamente il codice sorgente di un software senza però eseguirlo. Il grosso vantaggio di tale tecnica consiste nella sua applicazione alla radice del processo di sviluppo, in contrasto alle esistenti tecniche di testing che vedevano posticipata la correzione dei difetti alla fase di pre-rilascio. Anticipare l'identificazione del difetto software comporta minori costi di correzione; è proprio questo il motivo del successo dell'analisi statica.

\section{Storia}
La nascita delle tecniche di analisi statica viene solitamente attribuita al tool Lint, sviluppato da Stephen C. Johnson e rilasciato alla fine degli anni '70. Lint fu realizzato allo scopo di segnalare come sospetti alcuni costrutti nel sorgente in linguaggio C, come la mancanza di punti e virgola, parentesi, cast impliciti, ecc. 
Lint era integrato con il processo di compilazione, soluzione che sembrava essere la migliore per riportare segnalazioni relative al codice e che ne contribuì alla diffusione.\\
Purtroppo le limitate capacità di analisi, quali ad esempio l'obbligo di eseguire la scansione un file per volta, fecero si che Lint riportasse un elevata percentuale di rumore tra i risultati, ovvero valori corretti dal punto di vista dell'analisi ma irrilevanti per lo sviluppatore al fine di correggere difetti. Ciò si tradusse nella necessità di eseguire dei controlli manuali sui risultati di Lint, esattamente la situazione che Lint si era proposto di eliminare.
Per tale motivo Lint non fu mai adottato globalmente come tool per l'individuazione di difetti.\\
Nei primi anni 2000 una seconda generazione di tools emerse, che si è evoluta fino ad oggi. Gli sviluppatori intuirono che era necessario comprendere attraverso il software di analisi maggiori dettagli relativi al funzionamento del programma. Produssero tools in grado di  analizzare più files contemporaneamente e di identificare i percorsi di flusso dei dati, ma si scontrarono con la problematica che da sempre caratterizza l'analisi statica: il necessario compromesso da attuare tra performance ed accuratezza. L'efficacia delle tecniche di analisi statica è altamente condizionata dal fatto che devono essere gli sviluppatori ad utilizzarle, poichè prima si è in grado di identificare il difetto e minore costo ha la sua correzione.

\section{Applicazioni}
Sebbene le tecniche di analisi statica nacquero allo scopo di individuare difetti e di aderire a standard nella stesura del codice, molteplici successivi utilizzi vennero identificati ed implementati. \\
Attualmente l'analisi statica è utilizzata negli IDE\footnote{Integrated Development Environment} per evidenziare e per mostrare eventuali errori alla sintassi, per riportare segnalazioni in caso di non adesione agli standard di scrittura del codice in molteplici linguaggi.\\
In caso di software che possiede requisiti di sicurezza, ed ultimamente sempre più spesso visto il diffondersi di applicazioni critiche sotto questo punto di vista, l'analisi statica consente di individuare codice potenzialmente vulnerabile.

\section{Analisi statica vs. analisi dinamica vs. code review}
Le tecniche di analisi statica analizzano il sorgente di un programma in modo automatico senza eseguirlo. Prima della nascita di tali tecniche gli sviluppatori effettuavano un controllo manuale sul codice chiamato \emph{code review}.\\

La code review può essere eseguita da più sviluppatori (\emph{peer review}) oppure da un solo sviluppatore. E' una procedura complessa, che ha come requisito fondamentale la piena conoscenza delle decisioni architetturali prese durante la progettazione e la scrittura del codice oltre ad un'ottima padronanza del linguaggio in analisi. \\
Esistono due categorie di code review: \emph{formal code review} e \emph{lightweight code review}. La prima categoria richiede un dettagliato processo di analisi suddiviso in molteplici fasi. Tale metodologia comporta l'analisi di copie stampate del materiale ed è svolta da più partecipanti che contemporaneamente analizzano il codice.\\
La seconda categoria richiede meno formalismi rispetto alla precedente e viene svolta solitamente durante il normale processo di sviluppo. All'interno di essa si possono individuare le seguenti pratiche:
\begin{itemize}
\item Over-the-shoulder: uno sviluppatore osserva il codice che l'altro sta scrivendo per segnalare eventuali problemi
\item Email pass around: un SCM\footnote{Source Code Management System} invia tramite email il nuovo codice inserito nella codebase ad un soggetto che si occupa di effettuare la review.
\item Pair programming: Due sviluppatori scrivono codice contemporaneamente sulla stessa workstation. 
\item Tool-assisted code review: Sviluppatori e reviewers usano tools in grado di effettuare code review collaborativa.
\end{itemize}
La problematica di questa tipologia di analisi consiste nel tempo che richiede; i dati raccolti dai maggiori operatori del settore stimano che in media si possa effettuare code review su 150 linee di codice per ogni ora, fino a rimuovere l'85\% dei difetti presenti nel software.

L'analisi dinamica è una tecnica che consiste nell'osservare il comportamento del software durante la sua esecuzione. Al fine di rendere tale tecnica effettiva è necessario che il programma venga eseguito con diversi input. L'analisi dinamica è una tecnica precisa, che non comporta approssimazione poichè osserva l'esatto comportamento runtime dell'applicazione.
Lo svantaggio dell'analisi dinamica è la sua specificità: i risultati proposti riguardano solo ed esclusivamente quell'esecuzione, non c'è garanzia che la test suite utilizzata esegua effettivamente tutti i possibili data flow all'interno del software e che quindi esegua ogni porzione di codice. L'approccio definito dall'analisi dinamica è particolarmente adatto per il testing ed il debugging.

Analisi statica ed analisi dinamica sono due approcci complementari che possono essere applicati allo stesso problema, i risultati hanno però diverse proprietà e l'esecuzione di ognuno ha diversi costi. Per tale ragione esistono soluzioni in grado di combinare analisi statica ed analisi dinamica al fine di ridurre i difetti tipici delle due tecniche e fornire risultati più attendibili.

\chapter{Applicazione dell'analisi statica alla sicurezza di applicazioni web}
Cosa posso risolvere usando analisi statica delle vulnerabilità precedenti?

\chapter{Analisi statica di codice PHP}

\chapter{Comparazione dei principali tool esistenti}

\section{Pixy}

\section{Saner}

\section{RIPS}

\chapter{Vulture}

\section{Problematiche}

\section{Sviluppi futuri}

\chapter{Discussione}

\chapter{Conclusioni}

\chapter{Applicazione dell'analisi statica alla sicurezza di applicazioni web}

\begin{epigraphs}
\qitem{"If you spend more on coffee than on IT security, you will be hacked. What's more, you deserve to be hacked."}{---\textsc{ Richard Clarke, White House Cybersecurity Advisor}}
\end{epigraphs}

Analizzando i reports di più attacchi eseguiti con successo alla sicurezza di applicazioni web si è notato che molti di questi non sono rivolti alla scoperta di nuove tipologie di vulnerabilità, bensì alla ricerca di vulnerabilità note.\\ Sebbene sia facile imputare la continua presenza di tali vulnerabilità alla negligenza dello sviluppatore, il vero problema è dato dal fatto che le tecniche per evitare queste problematiche non sono codificate nel processo di sviluppo software e non è pensabile fare affidamento sulla memoria dello sviluppatore per evitarle. E' questo il motivo per cui occorrono dei sistemi in grado di rilevare tali vulnerabilità direttamente nel processo di sviluppo, ed è in questo caso che l'analisi statica entra in gioco.

I primi approcci all'analisi statica orientata alla sicurezza si possono trovare nei tools RATS\cite{rats}, ITS4\cite{its4} e Flawfinder\cite{flawfinder}, capaci di effettuare il parsing del codice sorgente al fine di trovare chiamate a funzioni pericolose. Dopo aver trasformato in tokens il codice sorgente (il primo step che anche il compilatore esegue), questi tools effettuavano una comparazione tra lo stream di tokens generato ed una libreria di costrutti vulnerabili. Sebbene questi approcci basati sull'analisi lessicale furono un passo avanti rispetto a grep, producevano un elevato numero di falsi positivi. Uno stream di tokens è certamente meglio di uno stream di caratteri, ma nemmeno questi tools avevano una minima conoscenza del funzionamento del programma. \\
Con il passare degli anni i tool per l'analisi statica orientata alla sicurezza sono cresciuti in complessità e sono diventati più sofisticati. La vera evoluzione si è avuta con la presa in considerazione del contesto delle chiamate e delle informazioni semantiche di un programma. Questo ha reso possibile l'individuazione delle condizioni entro cui una vulnerabilità si può manifestare, aumentando l'accuratezza e l'efficienza.\\
Costruendo un AST\footnote{Abstract Syntax Tree} dal codice sorgente questi tool possono effettivamente considerare la semantica del programma in esame. Dall'AST è possibile eseguire diverse tipologie di analisi:
\begin{itemize}
\item Local analysis: esamina il programma una funzione per volta e non considera le relazioni tra le funzioni.
\item Module-level analysis: considera un modulo/classe per volta ma non considera le chiamate tra i moduli.
\item Global-level analysis: considera il programma per intero prendendo in considerazione tutte le chiamate tra funzioni.
\end{itemize}
A seconda della tipologia di analisi dipende la quantità di contesto che il tool deve considerare, più contesto significa meno falsi positivi ma anche più computazione.\\
Un buon tool per l'analisi statica orientata alla sicurezza deve essere facile da usare ed i risultati devono essere comprensibili agli sviluppatori che non conoscono approfonditamente le problematiche di sicurezza. Per far si che un tool di analisi statica venga utilizzato come parte attiva nello sviluppo è importante che il tempo di computazione non sia troppo elevato viceversa rischia di essere utilizzato saltuariamente, fallendo il l'obiettivo per il quale è stato ideato.

\section{Tecniche di analisi}
In letteratura sono presenti diversi approcci per l'esecuzione di analisi statica finalizzata alla scoperta di vulnerabilità di sicurezza. Tutti sono però contraddistinti dalla seguenti tre fasi:
\begin{itemize}
\item Costruzione del modello
\item Analisi
\item Report dei risultati
\end{itemize}

\subsection{Costruzione del modello}
La fase di costruzione del modello riguarda l'astrazione che è necessario effettuare sul codice sorgente. Come esposto in precedenza esistono tool che lavorano direttamente sul codice, il più primitivo dei quali è grep, tuttavia lavorare direttamente su di esso è problematico poiché è necessario costruire espressioni regolari complesse e non si ha una visione d'insieme dell'intero programma e delle procedure.\\
Nell'analisi statica solitamente si pre-processa il codice per ottenere una rappresentazione ad un livello più basso, sfruttando le normali operazioni eseguite dal compilatore:
\begin{enumerate}
\item Line reconstruction: nei linguaggi che consentono spaziature arbitrarie tra gli identificativi è necessaria la presenza di una fase che ricostruisca le linee restituendo una forma canonica. In alcuni vecchi linguaggi questa fase eseguiva anche una normalizzazione nel caso venisse usato lo stropping\footnote{Tecnica che consiste nell'utilizzare una stringa sia come parola chiave che come identificatore.}.
\item Analisi lessicale: il codice viene pulito da elementi inutili come spaziature e commenti e trasformato in uno stream di tokens. Ogni token è una singola unità atomica del linguaggio, ad esempio una parola chiave, un identificativo oppure il nome di un simbolo. Tool come RATS, ITS4 e Flawfinder si limitavano ad utilizzare il risultato di questa fase come input per la propria analisi.
\item Analisi sintattica: lo stream di tokens viene trasformato in un \emph{parse tree}, ovvero una rappresentazione ad albero del codice sorgente. Viene così determinata la struttura dello stream di tokens in input grazie all'uso di una data grammatica formale. L'elemento che esegue questa operazione è detto \emph{parser}.
\item AST: il parse tree creato nella fase precedente viene trasformato in un \emph{abstract syntax tree}. Un AST è un albero simile al parse tree ma pulito di tutti i tokens non utili per l'analisi semantica.
\item Analisi semantica: Ad ogni token viene attribuito un significato al fine di creare una tabella dei simboli. Questa fase esegue il controllo dei tipi di dato (type checking), l'associazione tra le referenze delle funzioni e le loro definizioni (object binding), il controllo che tutte le variabili siano state inizializzate prima dell'uso (definite assignment).
\item Analisi del flusso di controllo: Tutti i possibili percorsi che possono essere eseguiti all'interno del codice vengono tradotti in una serie di \emph{Control flow graphs}. Il flusso di controllo tra le funzioni è raccolto nei \emph{call graphs}.
\item Analisi del flusso dei dati: L'analisi controlla come i dati si muovono all'interno del programma. Vengono a tale scopo usati i grafici realizzati nella fase precedente. Il compilatore usa tale analisi per allocare i registri, rimuovere codice non utilizzato ed ottimizzare l'uso di processore e memoria.
\end{enumerate}

Nonostante il compilatore esegua tutte queste fasi per l'interpretazione del codice sorgente, i tool di analisi statica generalmente si limitano ad alcune di queste operazioni prima di effettuare l'analisi. In base al dato che si analizza esistono innumerevoli approcci a tale fase, ognuno con le proprie caratteristiche e di conseguenza con risultati molto differenti tra loro.

\subsection{Analisi}
Esistono generalmente due approcci per realizzare l'analisi statica di un'applicazione: \emph{local analysis} o \emph{intraprocedural analysis} e \emph{global analysis} o \emph{interprocedural analysis}.

La local analysis si occupa di analizzare una funzione individualmente, tenendo traccia delle proprietà dei dati nella funzione e delle condizioni per le quali una funzione può essere chiamata. Tracciare tali proprietà diventa però problematico in caso di loop e branches, siccome è necessario salvare lo stato di tutti i dati in ogni momento dell'esecuzione la quantità degli stessi diventa ingente rendendo l'approccio poco praticabile. Riducendo la precisione tuttavia si possono ottenere comunque risultati apprezzabili con quantità di dati inferiori, astraendo le proprietà del programma che non sono di interesse (abstract interpretation) come può essere l'ordine in cui le istruzioni vengono eseguite (flow-insensitive analysis).

A seconda dello stato globale del programma una funzione può riportare risultati differenti. La global analysis consiste nel comprendere il contesto entro il quale una funzione viene eseguita, per valutare le implicazioni conseguenti. A tale scopo è possibile eseguire una \emph{whole-program analysis} che consiste nel combinare in un'unica funzione tutto il codice che viene eseguito sostituendo le chiamate a funzioni con i relativi metodi. Tale tecnica non viene spesso utilizzata poiché porta alla generazione di un unico blocco di codice sorgente molto lungo che richiede molta memoria e molto tempo per essere analizzato.\\
Una tecnica alternativa è chiamata \emph{function summaries} e consiste nel trasformare una funzione in un insieme di precondizioni e postcondizioni, usando la conoscenza ottenuta tramite la local analysis (talvolta infatti le due tecniche vengono combinate per affinare i risultati). Quando viene eseguita l'analisi si tengono in considerazione solo tali elementi per determinare gli effetti che la funzione ha sull'intero programma. 

Dopo aver determinato un'approccio all'analisi statica occorre definire delle policy affinché sia possibile rintracciare all'interno del flusso di istruzioni le eventuali vulnerabilità. Ogni vulnerabilità viene quindi analizzata singolarmente e vengono individuati gruppi di regole che ne permettono l'individuazione.\\
Pixy\cite{pixy}, un tool di analisi statica orientato alla scoperta di vulnerabilità, richiede che le regole vengano specificate attraverso un file di testo, che definisce le funzioni di sanitizzazione ed i sinks. Altri tool però lavorano in modo differente, basandosi sulle annotations (Splint\cite{splint} ad esempio), altri ancora accettano la definizione delle regole mediante l'uso di un linguaggio apposito (solitamente PQL\cite{pql}).

\subsection{Report dei risultati}
Al termine della fase di analisi è necessario che i risultati vengano processati per essere mostrati all'utente. Questa fase è molto importante, il valore di un software di analisi statica è infatti direttamente collegato alla sua capacità di riportare risultati in modo comprensibile ed immediato per l'utente.\\
Tra i possibili risultati di un'analisi figurano: falsi positivi, falsi negativi, veri positivi e veri negativi. I falsi positivi si verificano quando una parte del codice è segnalata impropriamente come vulnerabile, i falsi negativi quando non viene rilevata la presenza di una vulnerabilità esistente. La seconda categoria di falsi è maggiormente problematica poiché fornisce un falso senso di sicurezza. I veri positivi indicano una vulnerabilità correttamente identificata, i veri negativi indicano l'assenza di una segnalazione per una porzione di codice sicura.

In vari tool i risultati vengono categorizzati in base alla criticità, determinata tenendo conto dell'accuratezza con cui tale vulnerabilità viene rilevata, in altri si tiene conto della profondità, ovvero del numero di chiamate e di loop/branches da analizzare.\\
Talvolta insieme alla segnalazione viene mostrato un help contestuale che motiva all'utente la presenza della stessa, al fine di renderne maggiormente comprensibile la natura.

\chapter{Analisi statica di codice PHP}

\begin{epigraphs}
\qitem{"I was really, really bad at writing parsers. I still am really bad at writing parsers."}{---\textsc{ Rasmus Lerdorf, PHP author}}
\end{epigraphs}

\section{PHP}
L’8 giugno del 1995 con un messaggio su Usenet Rasmus Lerdorf annunciava la disponibilità di Personal Home Page Tools versione 1.0 (PHP Tools 1.0), la prima release ufficiale di PHP\cite{php}. Questo set di files scritti in C permetteva a Lerdorf di registrare le visite al proprio Curriculum Vitae senza per forza dover accedere alle statistiche del server.\\
Lerdorf decise di rilasciare PHP Tools sotto licenza GPL, allo scopo di organizzare un gruppo di utenti in grado di fare crescere la propria creazione. Pochi mesi dopo l’annuncio di Personal Home Page Tools, Lerdorf annunciò il rilascio di un parser di nome FI (form-interpreter) da lui stesso sviluppato allo scopo di far interagire le pagine web con mSQL (predecessore dell’attuale mySQL). PHP prese a quel punto il nome PHP/FI, ispirandosi all'acronimo TCP/IP.

L'idea di integrare mSQL all’interno delle pagine web fu indubbiamente ciò che contribuì alla rapida diffusione di PHP e permise la creazione di quel gruppo di sviluppatori che da alcuni mesi Lerdorf cercava di ampliare.
Con l’avvento della versione 2.0 il set di script PHP e il parser FI vennero completamente riscritti ed il progetto iniziò a diffondersi globalmente conquistando il traguardo della presenza sull’1\% dei domini web. I file di PHP 2.0 avevano estensione .phtml ed il parser FI poteva comunicare con più di una tipologia di database (mSQL, Postgres95 e il neonato mySQL).\\
La vera svolta nel progetto però avvenne alla fine del 1997 quando due israeliani (Zeev Suraski e Andi Gutmans) svilupparono Zend Engine, un nuovo parser che nel giro di 8 mesi sostituì il parser di PHP/FI 2.0. Nel 1993 venne rilasciato PHP 3, che oltre a segnare l’esplosione di PHP come linguaggio di scripting per il web, segnò la fine dell’ era Lerdorf all’interno del team di sviluppo. Infatti il creatore di PHP iniziò a defilarsi mentre nel team crescevano le personalità di Suraski e Gutmans. PHP 3 fu anche la release che cambiò il significato dell'acronimo PHP, non più Personal Home Page ma, PHP: Hypertext Preprocessor.\\
Nel 2000 venne rilasciata la quarta versione di PHP, con notevoli miglioramenti sotto il fronte delle API e della velocità di esecuzione. Un grosso cambiamento della versione 4 riguardò la licenza, GNU General Public License (GPL) venne sostituita da PHP4 License, maggiormente restrittiva sebbene sempre open source. Fu questa release a consolidare il ruolo di PHP nel mondo dei linguaggi di programmazione orientati al web. Quattro anni dopo fu rilasciato PHP 5, con un migliorato supporto alla programmazione ad oggetti e il nuovo supporto ai web services.\\
La versione attuale di PHP è la 5.3. In questa release il supporto agli oggetti è stato esteso con l'aggiunta dei namespace, ovvero un sistema che permette di raggruppare variabili, classi e funzioni all’interno di un determinato spazio dei nomi al fine da diminuire le possibilità di collisione. Inoltre risultano ora supportati i late static binding e le closures.\\
Nonostante questi miglioramenti il linguaggio vive di un contrasto perenne tra le community di sviluppatori. Ritenuto spesso male organizzato e poco evoluto rispetto ai rivali che via via si stanno affermando in ambito web come Python e Ruby, viene molto apprezzato per la facilità di deploy. A livello enterprise è importante ricordare che, sebbene attraverso un meccanismo che traduce PHP in C chiamato HipHop\cite{hiphop}, la rete sociale più grande del mondo\footnote{http://www.facebook.com} ha il proprio codice sorgente in linguaggio PHP.

\section{Caratteristiche del linguaggio}
PHP è un linguaggio di scripting, ovvero un linguaggio di programmazione interpretato. A differenza dei linguaggi di programmazione compilati, che compilano il proprio codice in linguaggio macchina prima dell'esecuzione, il codice PHP viene eseguito per mezzo di un interprete.\\
In ambito web viene utilizzato attraverso l'uso di un'estensione applicata al web server, per consentire la generazione dinamica di codice HTML. Per Apache, il web server più diffuso, l'estensione è chiamata mod\_php.\\
PHP possiede un gran numero di librerie per eseguire ogni tipo di operazione, dall'elaborazione di immagini alla manipolazione di documenti XML fino alla crittografia.\\
PHP non si basa su una specifica formale, l'unica documentazione della semantica del linguaggio è data dalla definizione nel codice sorgente dell'implementazione. Tale mancanza rende la modellazione del comportamento del programma in esame complessa per un tool di analisi.\\
Biggar e Gregg sono gli autori di phc\cite{phc}, un compilatore alternativo rispetto a Zend che supporta solo codice PHP 4. Durante lo studio della semantica di PHP si sono scontrati con la mancanza di un modello formale per tale linguaggio ed hanno riportato nel proprio studio\cite{biggar} le caratteristiche che ritenute ambigue e problematiche. Siccome l'analisi della semantica del linguaggio è necessaria anche nei tool di analisi statica, di seguito vengono riportate alcune problematiche da loro evidenziate:
\begin{itemize}
\item \emph{Incongruenze tra PHP 4 e PHP 5}: Le differenze implementative tra PHP 4 e PHP 5 rendono l'analisi statica del codice sorgente problematica. Un caso emblematico riguarda il passaggio di variabili negli argomenti di un metodo che in PHP 4 avveniva di default per valore, in PHP 5 per riferimento. Benchè codice PHP 4 sia ormai datato questa differenza è ancora un vincolo importante per l'esecuzione di un'analisi statica compatibile.
\item \emph{php.ini}: Il file di configurazione php.ini influisce sul programma in esame, ad esempio la direttiva \emph{include\_path} definisce i files che vengono automaticamente aggiunti alla codebase, mentre \emph{magic\_quotes\_gpc} cambia il modo secondo il quale le stringhe vengono gestite.
\item \emph{Release}: Le nuove release di PHP possono alterare la semantica del linguaggio anche se contengono solo bugfix.
\end{itemize}

Oltre a queste problematiche la dinamicità del linguaggio comporta la presenza di alcune situazioni difficili da esaminare con la sola analisi statica. Tra queste:
\begin{itemize}
\item \emph{Valutazione del codice run-time}: Il costrutto \emph{eval} consente di eseguire come istruzioni le sequenze di caratteri contenute in una stringa come se fossero codice sorgente. Il contenuto di tale stringa però non è conosciuto prima dell'esecuzione del programma, rendendo di fatto impossibile l'analisi di tale contenuto.
\item \emph{Inclusione run-time di files esterni}: In PHP è possibile definire l'inclusione di files a seconda dello stato di variabili valutate run-time. Il valore di tali variabili non è definibile tramite analisi statica.
\item \emph{Tipizzazione dinamica dei dati}: Il tipo di una variabile non necessita di essere dichiarato e può mutare in modo trasparente a run-time. Questa peculiarità del linguaggio rende complessa la tracciatura dei dati nel flusso di esecuzione poiché possono essere convertiti più volte in modo implicito.
\item \emph{Duck typing}: Valori in un oggetto possono essere aggiunti o rimossi in ogni momento dell'esecuzione, facendo si che non risulti possibile predire l'occupazione in memoria dell'oggetto dalla sua dichiarazione.
\item \emph{Aliasing}: E' possibile definire alias in grado di referenziare il valore di altre variabili. Tali alias sono mutabili e possono essere creati e distrutti run-time.
\item \emph{Variabili di variabili}: Il valore di una stringa può essere usato come indice di un'altra variabile. Ciò è possibile grazie all'uso di una tabella dei simboli al posto di rigide locazioni di memoria.
\end{itemize}

E' importante poi ricordare come esistano alcune caratteristiche presenti in pressoché tutti i linguaggi che rendono le tecniche di analisi statica molto complesse: 
\begin{itemize}
\item \emph{Funzioni per la manipolazione di stringhe}: la presenza di una libreria per la manipolazione di stringhe rende complessa l'analisi statica poiché è importante conoscere l'esatto significato di ogni funzione al fine di capire dove e come un valore tainted si può propagare.
\item \emph{Espressioni regolari}: la manipolazione di stringhe tramite espressioni regolari complica l'analisi. Teoricamente occorrerebbe comprendere il funzionamento dell'espressione regolare e di conseguenza capire se tale espressione è in grado di agire su valori tainted ed in che modo.
\end{itemize}

\section{Modalità di analisi di codice PHP}
In letteratura sono presenti diversi approcci all'analisi statica di codice PHP. Come riportato in precedenza, la maggiore differenza riguarda la struttura oggetto in input all'analisi, ovvero quali tipologie di istruzioni vengono processate per ottenere i risultati.\\
L'input per un tool di analisi statica non è altro che l'output di una delle varie fasi nel processo di interpretazione del codice:
\begin{itemize}
\item Codice sorgente
\item Lexical analysis
\item Syntax analysis
\item Bytecode generation
\end{itemize}
Il codice sorgente non è mai stato utilizzato efficacemente come input nel campo dell'analisi statica. Le regole lessicali che governano il linguaggio rendono estremamente complessa l'interpretazione del codice, che necessita di essere processato per mezzo di espressioni regolari.\\
Per tale motivo i primi tool che eseguivano analisi statica di codice PHP prendevano in input uno stream di tokens, ovvero una rappresentazione delle istruzioni indipendente dalle regole lessicali del linguaggio. Tale output è frutto della fase di lexical analysis nel processo di interpretazione del codice.
Si citano due librerie in grado di convertire da codice sorgente a stream di tokens, \emph{tokenizer}\cite{tokenizer}, inclusa nella release di PHP e \emph{PHP\_TokenStream}\cite{tokenstream}.

La prima libreria ha un output più semplice, composto da un array di tre elementi: token, numero di riga e corrispondenza in codice PHP, come mostrato nell'esempio sottostante (nel primo listing è riportato il codice sorgente, nel secondo la corrispondente rappresentazione sotto forma di tokens). \\

\begin{lstlisting}[language=PHP]
<?php 
function foo() {
	if (TRUE) { 
		$foo = TRUE;
	} else {
		$foo = FALSE;
	}
	return NULL;
}
?>
\end{lstlisting}

\begin{lstlisting}
Array (
	[0] => Array (
		[0] => 368			T_OPEN_TAG
		[1] => <?php\n		text 
		[2] => 1				line number
	)
	[1] => Array (
		[0] => 334			T_FUNCTION
		[1] => function 
		[2] => 2
	)
	[2] => Array (
		[0] => 371			T_WHITESPACE 
		[1] => 
		[2] => 2
	)
	[3] => Array (
		[0] => 307			T_STRING 
		[1] => foo 
		[2] => 2
	)
	[4] => ( 
		.
		.
		.
\end{lstlisting}

La seconda libreria invece fornisce come output una serie di tokens rappresentati come oggetti, riportati nell'esempio sottostante.\\

\begin{lstlisting}
PHP_Token_Stream Object (
	[flags:SplDoublyLinkedList:private] => 0 
	[dllist:SplDoublyLinkedList:private] => Array
	(
		[0] => PHP_Token_OPEN_TAG Object 
		(
			[id:protected] => 368
			[text:protected] => <?php\n
			[line:protected] => 1
		)
		.
		.
		
		[4] => PHP_Token_OPEN_BRACKET Object 
		(
			[id:protected] => 501
			[text:protected] => (
			[line:protected] => 2
		)
		.
		.
		
\end{lstlisting}

Al termine della conversione, i tool di analisi statica possono così operare su una struttura più semplice da parsare rispetto al puro codice sorgente.

La fase successiva nell'interpretazione di codice PHP consiste nella syntax analysis, ovvero nel parsing per ottenere un abstract syntax tree. Tre librerie sono state individuate per la realizzazione di tale fase: \emph{PHP\_Reflection\_AST}, \emph{parse\_tree}\cite{parsetree} e \emph{PHP-Parser}\cite{phpparser}. Le prime due risultano abbandonate da anni, non supportano quindi i costrutti delle più recenti release di PHP; l'ultima, sebbene rilasciata come alfa, è attualmente l'unica libreria scritta in PHP in grado di generare un abstract syntax tree di codice sorgente basato su PHP 5.3. Esiste poi un ulteriore promettente tool, chiamato \emph{PHP\_Depend}\cite{phpdepend}, che non si occupa di eseguire analisi statica del codice ma effettua analisi delle metriche. Al suo interno contiene una serie di librerie in grado di generare un AST parziale, derivate da \emph{PHP\_Reflection\_AST}.

L'ultima fase che produce risultati validi per un'analisi statica è la fase di bytecode generation. La libreria che si occupa di produrre tale output è \emph{Bytekit}\cite{bytekit} di Stefan Esser. Oltre a riportare una rappresentazione utente del bytecode generato in fase di compilazione, fornisce anche informazioni sul flusso di istruzioni in esecuzione. L'output di questa fase è però a basso livello e complesso da parsare durante un'analisi statica.

I tools analizzati nel documento corrente utilizzano come input valori provenienti dalle fasi di lexical analysis o syntax analysis, che forniscono una rappresentazione ad alto livello del flusso di istruzioni.

\chapter{Comparazione dei principali tool esistenti}
Nel seguente capitolo verranno esposte le modalità di funzionamento di alcuni tool che eseguono analisi statica di codice PHP al fine di trovare vulnerabilità. Verranno confrontati gli approcci e verrà valutata la possibile applicabilità durante il normale ciclo di sviluppo software. \\
Alcuni di questi tool sono effettivamente disponibili, altri sono dei prototipi accademici, altri ancora non supportano le versioni più recenti di PHP. l'obiettivo di tale sezione è quello di illustrare come la ricerca di vulnerabilità tramite analisi statica di codice PHP non presenti soluzioni affermate e si cercherà di spiegarne le motivazioni.

\section{Pixy}
Pixy è un tool di analisi statica scritto in Java per la detection di vulnerabilità in codice PHP 4, rilasciato sotto licenza GPL. Sviluppato da Jovanovic\ref{CITAZIONE} come lavoro accademico, inizialmente consentiva la ricerca di sole vulnerabilità di tipo XSS ma successivamente è stato esteso alla ricerca di altre vulnerabilità di tipo taint-style, come SQL injections e command injections.\\
Pixy è flow sensitive, ovvero prende in considerazione l'ordine delle istruzioni del programma e la scansione che effettua è di tipo globale (interprocedural analysis), ovvero non valuta una funzione singolarmente ma tiene in considerazione il contesto in cui viene eseguita. Inoltre Pixy è dotato di un meccanismo di analisi degli alias che consente di ridurre i numerosi falsi positivi che tale caratteristica del linguaggio potrebbe generare.\\
Pixy è disponibile anche in una versione web-based limitata sul sito ufficiale\footnote{http://pixybox.seclab.tuwien.ac.at/pixy/webinterface.php}.

La scansione di Pixy si basa sulla definizione di tre parametri:
\begin{itemize}
\item Gli entry points del programma: GET, POST e COOKIE.
\item Le funzioni di sanitizzazione: htmlentities, htmlspecialchars ed opportuno type casting.
\item Sensitive sinks: funzioni che ritornano output al browser, come print, echo, printf.
\end{itemize}
Queste definizioni sono stabilite tramite un file di configurazione in formato testuale.\\
L'obiettivo della scansione consiste nel determinare in quali circostanze è possibile che un dato tainted possa raggiungere un sensitive sink senza essere sanitizzato in modo opportuno. La tecnica per determinare ciò è quella della data-flow analysis, ovvero si computano in ogni punto del programma i possibili contenuti delle variabili per tracciare quali possono essere tainted. La data-flow analysis opera sul control flow graph del programma, quindi è necessario che prima dell'esecuzione venga costruito un albero (nello specifico un parse-tree) del file PHP in input. Tale operazione viene svolta con Jflex, un lexical analyzer scritto in java e Java Parser Cup, un generatore di parser in java.\\
Prima di eseguire la taint analysis sul control flow graph risultante vengono eseguite diverse operazioni, tra cui l'alias analysis (fornisce informazioni sugli alias) e la literal analysis (valuta le condizioni di branch ed esclude i percorsi che non possono essere eseguiti a run-time). Entrambe queste operazioni servono ad aumentare la precisione del tool, sebbene aumentino la complessità.\\
Pixy non considera le sanitizzazioni custom, generalmente effettuate con espressioni regolari, per due motivi: prima di tutto non è in grado di valutarle, inoltre si ritiene a priori che effettuare sanitizzazioni mediante l'uso di espressioni regolari sia una pessima decisione implementativa a causa della facilità di errore.

La grossa problematica che affligge Pixy è data dal fatto che non è in grado di parsare codice scritto per PHP 5. Le ingenti modifiche apportate da tale release al linguaggio hanno reso Pixy inadeguato per gli attuali progetti scritti in PHP.\\
Sebbene fosse molto promettente nel periodo in cui è stato sviluppato, il tool non riceve aggiornamenti dal 2007 e risulta effettivamente abbandonato. La mancanza di una solida community ha fatto si che nessuna terza parte abbia forkato il progetto per proseguirne lo sviluppo, rendendolo ormai troppo datato per l'utilizzo su codice sorgente attuale.

Secondo i risultati dei test eseguiti dall'autore[12 su paper depoel] Pixy ha una percentuale di falsi positivi che si aggira intorno al 50\%, soglia considerata dall'autore stesso accettabile per un tool di analisi statica su un linguaggio fortemente dinamico come PHP. Secondo le analisi eseguite da de Poel[CITAZONE] nel 2010, durante la scansione del codice sorgente di alcuni progetti open source viene dimostrato che Pixy non è in grado di funzionare con codebase recenti, la scansione ha riportato risultati per solo cinque su sedici dei software analizzati e, neppure in questi casi, i risultati sono stati attendibili.

\section{Saner}
Saner è un prototipo, basato su Pixy, che combina le peculiarità di un'analisi statica con quelle di un'analisi dinamica.
Sviluppato da Balzarotti et.al.\cite{CITAZIONE} dimostra come sia possibile combinare due approcci a prima vista molto differenti per migliorare i risultati ottenuti.\\
La parte di analisi statica viene eseguita da Pixy, che si occupa di determinare in che modo l'applicazione processa l'input, individuando eventuali sanitizzazioni incomplete. Successivamente interviene una fase di analisi dinamica che si occupa di ricostruire il codice responsabile per la sanitizzazione dell'input. Una volta ricostruito, Saner processa tale codice con input malevoli per individuare eventuali problemi nelle sanitizzazioni.\\
Facendo affidamento sull'analisi dinamica, Saner è in grado di valutare eventuali sanitizzazioni custom al posto di reputarle direttamente come inaffidabili, cosa che il solo Pixy era costretto a fare.\\
Nell'esempio seguente viene evidenziato il valore del contributo dell'analisi dinamica nella scelta della giusta decisione da intraprendere.

[ESEMPIO DA PAPER SANER]

La maggior parte dei tool di analisi statica avrebbero marchiato la prima parte del codice riportato come safe a causa della presenza del costrutto htmlentities, che solitamente fa parte delle funzioni di sanitizzazione. Nel secondo caso però è presente una sanitizzazione custom costruita con la funzione str\_replace. Siccome str\_replace non fa parte delle funzioni di sanitizzazione, il codice viene marchiato come unsafe ed un warning viene riportato all'utente. Saner consente di eseguire tale sanitizzazione in modo dinamico sfruttando input appositamente malevoli al fine di verificare l'efficacia della sanitizzazione stessa, riducendo il numero di warning.\\
In un certo senso, Saner automatizza le operazioni che uno sviluppatore dovrebbe fare per verificare se un warning riportato da un tool di analisi statica è effettivamente una problematica di sicurezza.

I risultati riportati dall'autore dimostrano come la fase di analisi dinamica aumenti in modo elevato l'affidabilità del tool. Nell'analisi di cinque applicativi open-source di media complessità, Pixy segnala l'incapacità di determinare il valore della procedura di sanitizzazione per ben 66 sinks. Con l'ausilio dell'analisi dinamica è stato possibile determinare l'effettiva presenza di 14 vulnerabilità, per i restanti 52 sinks l'analisi non è stata in grado di determinare dei valori di input che bypassassero la sanitizzazione.

\section{RIPS}
RIPS è un tool per l'analisi statica di codice PHP rivolto alla ricerca di vulnerabilità sviluppato da Johannes Dahse.
E' l'unico tool open source scritto in PHP considerato usabile attualmente in sviluppo.\\
Il tool è in grado di individuare vulnerabilità di tipo XSS, SQL Injection, file disclosure, code evaluation, remote command execution e busines logic flaw attraverso dei rule-sets definiti nei propri file di configurazione.\\
RIPS lavora in due fasi, costruzione del modello e analisi. Nella fase di costruzione del modello si possono individuare le operazioni di analisi semantica e lessicale, parsing e control flow analysis. Nella fase di analisi si individuano le operazione di taint analysis e local and global analysis.

La fase di analisi lessicale e semantica è caratterizzata dall'uso della libreria built-in \emph{tokenizer} per trasformare i costrutti del linguaggio in uno stream di tokens, attraverso l'uso della funzione \emph{token\_get\_all()} che trasforma ogni istruzione del linguaggio in un array costituito dal token, dal numero di riga dell'istruzione e dal codice originale. Successivamente vengono rimossi i tokens ritenuti inutili per l'analisi, come le spaziature ed il codice HTML, e trasformati alcuni tokens in equivalenti strutture più semplici da parsare (ad esempio vengono convertiti i costrutti di branches realizzati mediante la forma compatta in classici costrutti if-else).\\
La fase successiva, quella di parsing, avviene attraverso l'analisi dello stream di tokens una sola volta per ogni file, al fine di garantire le migliori performance. In questa fase viene creata per ogni file una lista delle dipendenze e per alcuni costrutti vengono eseguite operazioni particolari. In caso di \emph{T\_INCLUDE} ad esempio lo stream di tokens del file da includere viene aggiunto allo stream corrente, in caso di \emph{T\_VARIABLE} viene controllato lo scope della variabile, ovvero se è una variabile globale o locale, e di conseguenza viene aggiunta ad una lista.\\
La fase di control flow analysis avviene in modo estremamente diverso da Pixy: se nel tool di Jovanovic si faceva riferimento al control flow graph, in RIPS la generazione dello stesso è stata evitata per problemi di performance. RIPS esegue la control flow analysis sfruttando la presenza di alcuni tokens come le parentesi graffe, \emph{T\_EXIT} e \emph{T\_THROW} che danno indicazioni sulle uscite da flussi di istruzioni.

La fase di analisi si basa sulle direttive definite attraverso array PHP nei file di configurazione. Tali direttive indicano gli input, i sinks, le funzioni di sanitizzazione con i rispettivi parametri da tracciare ed i token da ignorare.\\
RIPS cerca chiamate a funzioni, controlla se la funzione trovata è nella lista dei sinks ed in caso affermativo valuta i parametri di tale funzione. Una volta individuati i parametri rilevanti controlla nelle liste precedentemente create di variabili globali e locali se tali possono essere o meno tainted. In caso affermativo viene riportata la potenziale vulnerabilità.\\
RIPS è in grado di determinare se una funzione definita dall'utente è o meno un sink. Nel caso in cui ciò accada infatti l'algoritmo può risalire ai parametri di input e quindi comprendere se il sink è determinato da uno di essi. In caso affermativo oltre alla funzione stessa anche le chiamate ad essa vengono trattate come un sink.

RIPS possiede una configurabile interfaccia web che consente di lanciare la scansione e che riporta i risultati. Tale interfaccia consente di definire per quali vulnerabilità effettuare la scansione, il livello di verbosity della scansione e se comprendere le sottodirectory o meno. Il report dei risultati è dettagliato, vengono mostrate le linee di codice sorgente che riportano problematiche,  

\chapter{Vulture}
Vulture è un concept/prototipo di tool per analisi statica volta alla ricerca di vulnerabilità di applicazioni web, creato dal sottoscritto in collaborazione con EURECOM.\\
Il progetto nasce con l'idea di costruire una piattaforma estensibile, alla quale lo sviluppatore può aggiungere i propri insiemi di regole volte alla ricerca di vulnerabilità. 

Vulture non è attualmente completo ed usabile, è stata implementata solo una parte delle caratteristiche progettate. 


\section{Problematiche}

\section{Sviluppi futuri}

\chapter{Discussione}

\chapter{Conclusioni}

%\appendix

%\chapter{Parti significative del codice dell'applicazione}
\section{Funzione principale dello scanner}
\begin{lstlisting}
/**
 *  Implementation of the crawler function.
 *  Flow: 
 *  1- set crawler_id but leave status = 0
 *  2- make status = 1
 *  3- make status = 2
 */
function crawler_framework($selection_query, $initial_status, $function_callback) {
  // Initialize the crawler
  db_query("INSERT INTO {crawler} VALUES (default)");
  $crawler_id = db_last_insert_id('crawler', 'id');
  $selection_query += array(
    'join' => '',
    'where' => '',
    'parameters' => array(),
  );
  $fields = empty($selection_query['fields']) ? '' : ','. implode(', ', $selection_query['fields']);
  $sql = 'SELECT l.path'. $fields .' FROM {crawler_links} AS l '. $selection_query['join'] .' WHERE crawler_id = %d and status = %d'. $selection_query['where'];
  array_unshift($selection_query['parameters'], $crawler_id, $initial_status+1);
  // Set execution time calc
  $time_old = time();
  // While max execution time is not reached and there's something to process inside crawler_links table (time() < $time_limit) && 
  while ((db_fetch_array(db_query_range("SELECT * FROM {crawler_links} WHERE status = %d", $initial_status, 0, 1)) != NULL)) {
    $status = $initial_status;
    //Mark the extracted page as visited
    $status++;
    db_query("UPDATE {crawler_links} SET crawler_id = %d, status = %d WHERE status = %d LIMIT 1", $crawler_id, $status, $initial_status);
    // Get the link from crawler_links table
    $selected_results = db_fetch_array(db_query_range($sql, $selection_query['parameters'], 0, 1));
    // Update the status field to sign as executed that link
    // (The following two lines could be move to the end of the function i think without problems) 
    db_query("UPDATE {crawler_links} SET status = status + 1 WHERE status = %d AND crawler_id = %d", $status, $crawler_id);
    $status++;
    // Create a new object and parse the page
    $obj = new drupal_security_scanner_test();
    // Set the cookie
    $session_cookie = variable_get('security_scanner_cookie', '');
    $obj->curl_options = array(
      CURLOPT_COOKIE => $session_cookie,
      CURLOPT_USERAGENT => 'security_scanner',
    );
    $obj->drupalGet($selected_results['path']);
    $obj->parse();
    $function_callback($obj, $selected_results);
    $obj->curlClose();
  }
  echo "<h2>Process Finished!</h2><br />";
  $execution_time = time() - $time_old;
  echo "<h4>Execution time: ". $execution_time ."seconds.</h4><br />";
  // Get the location of the website and put a link for that
  $exploded = explode("/", $_SERVER['PHP_SELF']);
  $url = array_slice($exploded, 0, -1);
  $url = implode("/", $url);
  echo "<h3>Back to your <a href=\"". $url ."\">Drupal WebSite</a></h3>";
}
\end{lstlisting}
\section{Funzione di callback per il crawling del sito}
\begin{lstlisting}
/**
 *  Crawler: page processing function
 */     
function security_scanner_page_processing($obj, $selected_results) {
  global $base_url;
  $links = $obj->elements->xpath('//a');
  foreach ($links as $link) {
    $url_to_save = (string)$link->attributes()->href;
    $absolute = getAbsoluteUrl($url_to_save);
    // Get the page but check:
    // a - if it's logout link, that makes me lose the cookie.
    // b - if it's security scanner, skip
    // c - if it's xss_injector, skip. That will launch the crawler
    // d - if it's cron.php, skip. that will make a loop
    $parsed_url = parse_url($absolute);
    $file = basename($absolute);
    if (!isset($parsed_url['query']))
      $parsed_url['query'] = '';
    if (($parsed_url['query'] != 'q=logout') && ($parsed_url['query'] != 'q=admin/settings/security_scanner') && ($parsed_url['query'] != 'q=admin/reports/updates') && ($parsed_url['query'] != 'q=admin/settings/xss_injector') && ($file != 'cron.php')) {  
      if (substr($absolute, 0, strlen($base_url)) == $base_url) {
        // Here we use IGNORE to insert only one time a link into the table. ("path" is a unique index)
        db_query("INSERT IGNORE INTO {crawler_links} (id, path, crawler_id, status) VALUES ('','%s','','')", $absolute);
      }
    }
  }
  // Get the forms inside the page
  $inputs = $obj->elements->xpath("//input[@name='form_id']");
  foreach ($inputs as $input) {
    $form_id = (string)$input->attributes()->id;
    // Debug line! HAS TO BE REMOVED
    // echo $form_id."Form inserted! <br />";
    // Here we use again IGNORE to insert only one time a form_id into the table. ("form_id" is the primary key)
    db_query("INSERT IGNORE INTO {crawler_forms} VALUES ('%s','%d')", $form_id, $selected_results['id']);
  }
}
\end{lstlisting}
\section{Funzione di callback per seeding del sito}
\begin{lstlisting}
/**
 *  xss_injector_page_processing: This function is called from the crawler_framework() with a callback.
 *  It executes the operations on the page that is visited.   
 */
function xss_injector_seeding_process($obj, $selected_results) {
  // Selecting only textareas and input type = 'text' into the form with form_id specified before seeding
  $xpath = "//input[@name='form_id' and @type='hidden' and @id='". $selected_results['id'] ."']//..//input[@type='text']|//textarea";
  $all_inputs = $obj->elements->xpath($xpath);
  // If I find a field textarea or an input type = text field i seed, otherwise i skip the seeding process
  if (!empty($all_inputs)) {
    foreach ($all_inputs as $input) {
      $form_id = $selected_results['id'];
      $name = (string)$input->attributes()->name;
      $pattern = variable_get('security_scanner_pattern', '<script>alert(\'xss\');</script>') . $form_id;
      $edit[$name] = $pattern;
    }
    $obj->drupalPost($selected_results['path'], $edit, TRUE);
  }
}
\end{lstlisting}
\section{Funzione di callback per la ricerca di pattern eseguiti}
\begin{lstlisting}
/**
 *  xss_injector_find_seeds() is a function called back from the crawler_framework
 *  to enable a search into the pages finding seeds
 */
function xss_injector_find_seeds($obj, $selected_results) {
  // I cannot have the form_id here, what i seeded could be on any page, so i have to loop
  // on every form_id and search into the page if this appears.
  $forms_id = db_query("SELECT id FROM {crawler_forms}");
  while ($form_id_array = db_fetch_array($forms_id)) {
    $form_id = $form_id_array['id'];
    $pattern = variable_get('security_scanner_pattern', '<script>alert(\'xss\');</script>') . $form_id;
    $position = strpos($obj->_content, $pattern);
    // If I find an occurrence of this it means that the xss injection was succesful, so report the error!
    if (!empty($position)) {
      // Save the seeds founded into the database if not exists
      $discovered_xss = array("form_id" => $form_id, "executed_on_page" => $selected_results['path']);
      $seeds_founds = variable_get('security_scanner_unprotected_forms', array());
      $signal = TRUE;
      foreach ($seeds_founds as $value) {
        if ($value['form_id'] == $form_id)
          $signal = FALSE;
      }
      if ($signal == TRUE) {
        $seeds_founds[] = $discovered_xss;
        variable_set('security_scanner_unprotected_forms', $seeds_founds);
      }
    }
  }
}
\end{lstlisting}

%%%%%%%%%%%%%%%%%%%%%%%%%%%%%%%%%%%%%%%%%%%%%%%%%%%%%%%%%%%%%%%%%%%%%%%%%
%% 
%% Bibliografia attraverso BIBTEX: possibili stili per la biblografia
%%
%% Da: http://amath.colorado.edu/documentation/LaTeX/basics/steps/bibstyles.html
%%
%% plain: Entries are ordered alphabetically
%%
%% unsrt: Entries are not ordered alphabetically, but in the order they are first referenced.
%%
%% abbrv: The bibliography looks the same as for plain style except that first names and names 
%%        of journals and months are abbreviated;
%%
%% alpha: The bibliography looks the same as for plain style except that the reference markers 
%%        are not just 1,2,3... but are based on authors' initials and publication year;
%% 
\addcontentsline{toc}{chapter}{Bibliografia} 
\bibliographystyle{plain}
\bibliography{bibliografia}
\nocite{*}

\end{document}