\chapter{Introduzione}
Solo 15 anni fa nessuno avrebbe mai pensato che Internet sarebbe diventata parte integrante della vita quotidiana di ogni individuo. Da semplice meccanismo di condivisione delle informazioni mediante pagine statiche, la rete si è evoluta fino a diventare una realtà dinamica ed interattiva, in grado di interagire con l'utente rendendo disponibile una quantità di informazioni immensa.\\
Il merito di tale risultato è senza dubbio da attribuire alla quantità di applicazioni web disponibili, da enciclopedie costruite in modo collaborativo a sistemi per gestire il proprio patrimonio economico da remoto, da soluzioni per acquistare e vendere oggetti a sistemi per condividere fotografie e per gestire la propria rete sociale. Non è possibile elencare tutte le soluzioni che la rete fornisce ma si può benissimo ammettere che per moltissime attività prima svolte esclusivamente offline, ora esiste un'alternativa tramite internet.

Le applicazioni web hanno però introdotto una nuova importante problematica: la sicurezza dei dati sensibili degli utenti. Diverse tipologie di vulnerabilità possono portare alla perdita di tali dati, a modifiche nel funzionamento dell'applicazione ed alla compromissione del server sul quale essa risiede. Sebbene tale rischi siano ingenti, per molteplici cause quali il rispetto delle tempistiche, l'assenza di adeguate conoscenze di sicurezza e di programmazione, questi vengono spesso sottostimati dagli sviluppatori e la presenza di vulnerabilità risulta essere tutt'altro che rara.

Al fine di ridurre il rischio, le applicazioni web vengono riviste manualmente da team di esperti in grado di identificarne le vulnerabilità in una fase successiva a quella di sviluppo, chiamata code review. La code review è un processo lungo e con alto margine di errore, spesso condizionato dal budget e dalle tempistiche.\\
Vi è quindi l'esigenza di strumenti in grado di rendere più efficiente il processo di code review, tra i quali è di fondamentale importanza l'analisi statica. 

L'analisi statica si è diffusa negli anni '70 con l'avvento di Lint, un tool rivolto alla ricerca automatica di problematiche nel codice sorgente di programmi scritti in C. Questa tecnica ha attualmente molteplici scopi: viene utilizzata per l'evidenziazione della sintassi e per il rispetto degli standard di scrittura codice negli IDE, per la generazione della documentazione e per trovare eventuali problematiche di sicurezza nel codice sorgente.
Proprio in quest'ultimo campo l'analisi statica può rivelarsi una scelta vincente, fornendo uno strumento alternativo alla review del codice, più impreciso ma decisamente più pratico e veloce. 

Nel lavoro di tesi vengono esposte in dettaglio le modalità secondo le quali l'analisi statica volta alla ricerca di vulnerabilità può integrarsi nel ciclo di sviluppo software. Viene dimostrato che l'uso dell'analisi statica durante lo sviluppo può rendere più economica la correzione di eventuali problematiche rispetto all'analisi a posteriori e può annullare la finestra di esposizione dovuta ad eventuali vulnerabilità. 

Al fine di presentare degli esempi concreti si è scelto il linguaggio PHP mediante il criterio della diffusione, con lo scopo di studiarne le peculiarità nella fase di interpretazione del codice. Tale studio ha consentito l'individuazione di punti di accesso attraverso i quali implementare meccanismi di analisi statica, confermati nella successiva analisi degli approcci implementati dagli strumenti open source esistenti, quali Pixy, Saner e RIPS. \\
In letteratura sono presenti numerosi studi\cite{depoel} che si occupano di analizzare in dettaglio i risultati che i vari strumenti di analisi statica riportano su determinate codebase, tuttavia questi effettuano un'analisi a scatola chiusa, limitandosi a trarre conclusioni in base all'esito della scansione, al numero di falsi ed al tempo impiegato.\\
In questa tesi si vogliono invece analizzare le caratteristiche dell'implementazione, determinando quali di queste hanno influito sui risultati e quali sono risultate superflue. Inoltre si osserva la scarsa adozione dei tool esistenti, che risulta essere in contrasto con quanto esposto fino a quel momento nella tesi e si cerca di fornire delle motivazioni valide per spiegare tale fenomeno.

Viene infine mostrato Vulture, un tool attualmente in sviluppo progettato in collaborazione con EURECOM nell'ambito di una ricerca sul rapporto tra l'analisi statica e le vulnerabilità di tipo Http Parameter Pollution, la cui implementazione è iniziata nel corso di questo lavoro di tesi. Vulture nasce con l'obiettivo di automatizzare il processo di scoperta di potenziali vulnerabilità nel codice sorgente al fine di fornire allo sviluppatore un feedback sufficientemente veloce da consentire l'uso del tool durante lo sviluppo software, garantendo una sorta di continuous integration\cite{fowler} rivolta all'analisi di sicurezza. Il lavoro relativo a Vulture effettuato nel corso di questa tesi è soprattutto di progettazione, ovvero la scelta delle caratteristiche implementative adatte alla realizzazione del tool in base alle analisi effettuate in precedenza su tool esistenti ed in base alle caratteristiche dell'interprete PHP.
 
Negli ultimi capitoli della tesi vengono infine discusse le problematiche individuate nell'ambito dell'analisi statica e si cerca di capire lo stato attuale del settore, se è maturo e competitivo oppure se esistono margini per un nuovo business che approcci le esigenze moderne delle software house.