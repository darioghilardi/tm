\chapter{Introduzione}
Solo 15 anni fa nessuno avrebbe mai pensato che Internet sarebbe diventata parte indispensabile per la vita quotidiana di ogni individuo. Da semplice meccanismo di condivisione delle informazioni mediante pagine statiche, la rete si è evoluta fino a diventare una realtà dinamica ed interattiva, in grado di interagire con l'utente.
Il merito di tale risultato è senza dubbio da attribuire alla quantità di applicazioni disponibili: da sistemi per gestire il proprio patrimonio economico da remoto, a sistemi per acquistare e vendere oggetti, a sistemi per condividere fotografie e per gestire la propria rete sociale.\\
Le applicazioni web hanno però introdotto una nuova importante problematica: la sicurezza dei dati sensibili degli utenti. Diverse tipologie di vulnerabilità possono portare alla perdita di tali dati, a modifiche nel funzionamento dell'applicazione ed alla compromissione del server sul quale essa risiede. Sebbene tale rischi siano ingenti, per molteplici cause quali il rispetto delle tempistiche, l'assenza di adeguate conoscenze di sicurezza e di programmazione, questi vengono spesso sottostimati dagli sviluppatori e la presenza di vulnerabilità non è rara. \\
Al fine di ridurre il rischio, le applicazioni web vengono spesso riviste manualmente da team di esperti in grado di identificarne le vulnerabilità. Spesso capita che i limiti imposti dal budget portino ad una riduzione nell'accuratezza dell'operazione di review, compito di per se lungo e con alto margine di errore.\\
Esistono però alcuni strumenti in grado di fornire soccorso a chi esegue la review del codice, incrementando la velocità del processo e riducendone di conseguenza i costi. Tali strumenti si occupano di eseguire un'analisi statica del codice sorgente dell'applicazione, ovvero verificano determinate proprietà senza che il codice venga eseguito.\\
In questo lavoro di tesi vengono analizzate le metodologie utilizzate per l'analisi statica, con vantaggi, svantaggi e con particolare approfondimento sull'implementazione di alcuni strumenti rivolti all'analisi di codice in linguaggio PHP.\\
Viene poi in seguito introdotto Vulture, un tool sperimentale per l'esecuzione di analisi statica di codice PHP. Vulture ha l'obiettivo di automatizzare il processo di scoperta di potenziali vulnerabilità nel codice sorgente al fine di fornire allo sviluppatore un report che semplifica l'operazione di review. Verranno discusse le decisioni architetturali prese durante la realizzazione del tool e le problematiche incontrate in fase di realizzazione. 