\chapter{Introduzione}
Solo 15 anni fa nessuno avrebbe mai pensato che Internet sarebbe diventata parte integrante della vita quotidiana di ogni individuo. Da semplice meccanismo di condivisione delle informazioni mediante pagine statiche, la rete si è evoluta fino a diventare una realtà dinamica ed interattiva, in grado di interagire con l'utente rendendo disponibile una quantità di informazioni immensa.\\
Il merito di tale risultato è senza dubbio da attribuire alla quantità di applicazioni web disponibili, da enciclopedie costruite in modo collaborativo a sistemi per gestire il proprio patrimonio economico da remoto, da soluzioni per acquistare e vendere oggetti a sistemi per condividere fotografie e per gestire la propria rete sociale. Non è di certo possibile elencare tutte le soluzioni che la rete ha fornito, ma si può benissimo ammettere che per moltissime attività che prima venivano svolte offline, ora esiste un'alternativa tramite internet.

Le applicazioni web hanno però introdotto una nuova importante problematica: la sicurezza dei dati sensibili degli utenti. Diverse tipologie di vulnerabilità possono portare alla perdita di tali dati, a modifiche nel funzionamento dell'applicazione ed alla compromissione del server sul quale essa risiede. Sebbene tale rischi siano ingenti, per molteplici cause quali il rispetto delle tempistiche, l'assenza di adeguate conoscenze di sicurezza e di programmazione, questi vengono spesso sottostimati dagli sviluppatori e la presenza di vulnerabilità risulta essere tutt'altro che rara.

Al fine di ridurre il rischio, le applicazioni web vengono riviste manualmente da team di esperti in grado di identificarne le vulnerabilità. Spesso capita che i limiti imposti dal budget portino ad una riduzione nell'accuratezza dell'operazione di review, compito lungo e che presenta un alto margine di errore.\\
Esistono però alcuni strumenti in grado di fornire soccorso a chi esegue la review del codice, tra cui è di fondamentale importanza l'analisi statica.

L'analisi statica si è diffusa negli anni '70 con l'avvento di Lint, un tool rivolto alla ricerca automatica di problematiche nel codice sorgente di programmi scritti in C, questa tecnica ha attualmente molteplici scopi: viene utilizzata per l'evidenziazione della sintassi e per il rispetto degli standard di scrittura codice negli IDE, per la generazione della documentazione e per trovare eventuali problematiche di sicurezza nel codice sorgente.
L'analisi statica ha la capacità di integrarsi nel ciclo di sviluppo software al fine di annullare la finestra di esposizione dovuta ad eventuali vulnerabilità e nel contempo ridurre i costi di gestione. Essa può rivelarsi vincente in questo campo, fornendo uno strumento alternativo alla review del codice, più impreciso ma decisamente più pratico e veloce.

Nel lavoro di tesi vengono esposte in dettaglio le modalità secondo le quali l'analisi statica volta alla ricerca di vulnerabilità può integrarsi nel ciclo di sviluppo software. Vengono studiate le peculiarità del linguaggio PHP ed in che modo è possibile implementare meccanismi per l'analisi statica di programmi realizzati in tale linguaggio. Si analizzano alcuni tool esistenti, mostrandone i dettagli implementativi, l'efficacia ed il possibile valore derivante dal loro uso in produzione.

Viene poi mostrato Vulture, un tool attualmente in sviluppo progettato in collaborazione con EURECOM nell'ambito di una ricerca sul rapporto tra l'analisi statica e le vulnerabilità di tipo Http Parameter Pollution, la cui implementazione è iniziata nel corso di questo lavoro di tesi. Vulture ha l'obiettivo di automatizzare il processo di scoperta di potenziali vulnerabilità nel codice sorgente al fine di fornire allo sviluppatore un report che semplifica l'operazione di review. Verranno discusse le decisioni architetturali prese durante la realizzazione del tool e le problematiche incontrate in fase di realizzazione.