\chapter{Conclusioni}
In questo documento sono state analizzate diverse tecniche per violare la privacy degli utenti sfruttando due tipologie di applicativi molto utilizzati: i social networks ed i file hosting services. \\
Per ognuna sono state evidenziate le modalità tramite le quali un attaccante può prendere possesso di dati sensibili ed è stato dimostrato come la fiducia degli utenti non venga poi ripagata dall'effettivo grado di protezione dei loro dati sensibili. Infatti si è dimostrato che i social networks ed i file hosting services, pur conoscendo tali problematiche, non sempre si sono dimostrati affidabili nel fornire l'adeguata protezione, per le ragioni più varie: commerciali, relative alla user interface, di convenienza.\\
Si è dimostrato poi come talvolta la fiducia e la non consapevolezza delle potenziali problematiche porti l'utente a compiere azioni in grado di mettere a repentaglio i propri dati personali.