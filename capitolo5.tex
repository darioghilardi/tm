\chapter{Analisi statica di codice PHP}

\section{PHP}
L’8 giugno del 1995 con un messaggio su Usenet Rasmus Lerdorf annunciava la disponibilità di Personal Home Page Tools versione 1.0 (PHP Tools 1.0), la prima release ufficiale di PHP. Questo set di files scritti in C permetteva a Lerdorf di registrare le visite al proprio Curriculum Vitae senza per forza dover accedere alle statistiche del server.\\
Lerdorf decise di rilasciare PHP Tools sotto licenza GPL, allo scopo di organizzare un gruppo di utenti in grado di fare crescere la propria creazione. Pochi mesi dopo l’annuncio di Personal Home Page Tools, Lerdorf annunciò il rilascio di un parser di nome FI (form-interpreter) da lui stesso sviluppato allo scopo di far interagire le pagine web con mSQL (predecessore dell’attuale mySQL). PHP prese a quel punto il nome PHP/FI, ispirandosi all'acronimo TCP/IP.

L'idea di integrare mSQL all’interno delle pagine web fu indubbiamente ciò che contribuì alla rapida diffusione di PHP e permise la creazione di quel gruppo di sviluppatori che da alcuni mesi Lerdorf cercava di ampliare.
Con l’avvento della versione 2.0 il set di script PHP e il parser FI vennero completamente riscritti ed il progetto iniziò a diffondersi globalmente conquistando il traguardo della presenza sull’1\% dei domini web. I file di PHP 2.0 avevano estensione .phtml ed il parser FI poteva comunicare con più di una tipologia di database (mSQL, Postgres95 e il neonato mySQL).\\
La vera svolta nel progetto però avvenne alla fine del 1997 quando due israeliani (Zeev Suraski e Andi Gutmans) svilupparono Zend Engine, un nuovo parser che nel giro di 8 mesi sostituì il parser di PHP/FI 2.0. Nel 1993 venne rilasciato PHP 3, che oltre a segnare l’esplosione di PHP come linguaggio di scripting per il web, segnò la fine dell’ era Lerdorf all’interno del team di sviluppo. Infatti il creatore di PHP iniziò a defilarsi mentre nel team crescevano le personalità di Suraski e Gutmans. PHP 3 fu anche la release che cambiò il significato dell'acronimo PHP, non più Personal Home Page ma, PHP: Hypertext Preprocessor.\\
Nel 2000 venne rilasciata la quarta versione di PHP, con notevoli miglioramenti sotto il fronte delle API e della velocità di esecuzione. Un grosso cambiamento della versione 4 riguardò la licenza, GNU General Public License (GPL) venne sostituita da PHP4 License, maggiormente restrittiva sebbene sempre open source. Fu questa release a consolidare il ruolo di PHP nel mondo dei linguaggi di programmazione orientati al web. Quattro anni dopo fu rilasciato PHP 5, con un migliorato supporto alla programmazione ad oggetti e il nuovo supporto ai web services.\\
La versione attuale di PHP è la 5.3. In questa release il supporto agli oggetti è stato esteso con l'aggiunta dei namespace, ovvero un sistema che permette di raggruppare variabili, classi e funzioni all’interno di un determinato spazio dei nomi al fine da diminuire le possibilità di collisione. Inoltre risultano ora supportati i late static binding e le closures.\\
Nonostante questi miglioramenti il linguaggio vive di un contrasto perenne tra le community di sviluppatori. Ritenuto spesso male organizzato e poco evoluto rispetto ai rivali che via via si stanno affermando in ambito web come Python e Ruby, viene molto apprezzato per la facilità di deploy. A livello enterprise è importante ricordare che, sebbene attraverso un meccanismo che traduce PHP in C chiamato HipHop, la rete sociale più grande del mondo\footnote{http://www.facebook.com} ha il proprio codice sorgente in linguaggio PHP.

\section{Caratteristiche del linguaggio}
Php è un linguaggio di scripting, ovvero un linguaggio di programmazione interpretato. A differenza dei linguaggi di programmazione compilati, che compilano il proprio codice in linguaggio macchina prima dell'esecuzione, PHP viene eseguito per mezzo di un interprete.\\
In ambito web viene utilizzato attraverso l'uso di un'estensione applicata al web server, per consentire la generazione dinamica di codice HTML. Per Apache, il web server più diffuso, l'estensione è chiamata mod\_php.\\
PHP possiede un gran numero di librerie per eseguire ogni tipo di operazione, dall'elaborazione di immagini alla manipolazione di documenti XML fino alla crittografia.\\
PHP non si basa su una specifica formale, l'unica documentazione della semantica è data dalla definizione del linguaggio nel codice sorgente della sua implementazione. Tale mancanza rende complessa per un tool di analisi la modellazione del comportamento del programma in esame.\\
Biggar e Gregg\cite{CITAZIONE DI PHC PAPER} sono gli autori di phc, un compilatore alternativo rispetto a Zend che supporta solo codice PHP 4. Durante lo studio della semantica di PHP hanno avuto a che fare con la mancanza di un modello formale ed hanno riportato nel proprio studio le caratteristiche che hanno ritenuto ambigue e problematiche. Siccome l'analisi della semantica del linguaggio è necessaria anche nei tool di analisi statica, di seguito vengono riportate alcune problematiche da loro evidenziate:
\begin{itemize}
\item Incongruenze tra PHP 4 e PHP 5: Le differenze implementative tra PHP 4 e PHP 5 rendono l'analisi statica del codice sorgente problematica. Un caso emblematico riguarda il passaggio di valori degli argomenti di un metodo che in PHP 4 avveniva di default per valore, in PHP 5 per riferimento. Benchè codice PHP 4 sia ormai datato questa differenza è ancora un vincolo importante per l'esecuzione di un'analisi statica efficiente.
\item Il file di configurazione php.ini influisce sul programma in esame, ad esempio la direttiva \emph{include\_path} definisce i files che vengono automaticamente aggiunti alla codebase, mentre \emph{magic\_quotes\_gpc} cambia il modo secondo il quale le stringhe vengono gestite.
\item Le nuove release di PHP possono alterare la semantica del linguaggio anche se contengono solo bugfix.
\end{itemize}

Oltre a queste problematiche la dinamicità del linguaggio comporta la presenza di alcune situazioni difficili da esaminare con la sola analisi statica. Tra queste:
\begin{itemize}
\item Valutazione del codice run-time: Il costrutto \emph{eval} consente di eseguire come istruzioni le sequenze di caratteri contenute in una stringa come se fossero codice sorgente. Il contenuto di tale stringa però non è conosciuto prima dell'esecuzione del programma, rendendo di fatto impossibile l'analisi di tale contenuto.
\item Inclusione run-time di files esterni: In PHP è possibile definire l'inclusione di files a seconda dello stato di variabili valutate run-time. Il valore di tali variabili non è definibile tramite analisi statica.
\item 
\end{itemize}

\chapter{Comparazione dei principali tool esistenti}

\section{Pixy}

\section{Saner}

\section{RIPS}

\chapter{Vulture}

\section{Problematiche}

\section{Sviluppi futuri}

\chapter{Discussione}

\chapter{Conclusioni}