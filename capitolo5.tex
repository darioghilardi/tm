\chapter{Analisi statica di codice PHP}

\section{PHP}
L’8 giugno del 1995 con un messaggio su Usenet Rasmus Lerdorf annunciava la disponibilità di Personal Home Page Tools versione 1.0 (PHP Tools 1.0), la prima release ufficiale di PHP. Questo set di files scritti in C permetteva a Lerdorf di registrare le visite al proprio Curriculum Vitae senza per forza dover accedere alle statistiche del server.\\
Lerdorf decise di rilasciare il suo set di tools sotto licenza GPL, allo scopo di creare un gruppo di utenti in grado di portare avanti la sua creazione. Pochi mesi dopo l’annuncio di Personal Home Page Tools, Lerdorf annunciò un parser di nome FI (form-interpreter) da lui stesso sviluppato allo scopo di far interagire le pagine web con mSQL (predecessore dell’attuale mySQL). PHP prese a quel punto il nome PHP/FI, ispirandosi all'acronimo TCP/IP.

La mossa di integrare mSQL all’interno di pagine web fu indubbiamente ci`o che rese PHP ancora piu` famoso e permise la creazione di quel gruppo di sviluppatori che da alcuni mesi Lerdorf cercava di ampliare.

\chapter{Comparazione dei principali tool esistenti}

\section{Pixy}

\section{Saner}

\section{RIPS}

\chapter{Vulture}

\section{Problematiche}

\section{Sviluppi futuri}

\chapter{Discussione}

\chapter{Conclusioni}