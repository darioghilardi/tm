\chapter{Strategie economiche}
\section{Il costo di usare soluzioni free}
E' inevitabile in un documento riguardante la sicurezza nei sistemi operativi Linux la trattazione degli eventuali vantaggi economici che possono derivare dall'adozione di questo genere di soluzioni in confronto alle soluzioni commerciali.\\
E' essenziale considerare il fatto che, nonostante sia possibile avere a disposizione gratuitamente lo strumento in grado di fornire una soluzione di sicurezza, esistono comunque costi impliciti dovuti all'implementazione dello strumento nel contesto, al training ed alle eventuali consulenze. \\
Supponendo per esempio di adottare un firewall software gratuito installato su una macchina con sistema operativo Linux, non avremmo costi per quanto riguarda l'acquisizione di licenze, o l'acquisto del software. Confrontando questa soluzione con un firewall hardware in vendita a costi irrisori dovremmo considerare il fatto che la soluzione commerciale ha un costo iniziale, ma non appena si è affrontato quel costo la soluzione è pronta per essere messa in funzione, senza costi aggiuntivi. Viceversa la soluzione basata su software libero necessita di configurazione dello strumento software.\\
Allo stesso modo un fattore da considerare con attenzione è il costo di manutenzione, una soluzione commerciale può avere costi di manutenzione molto variabili a seconda della soluzione scelta, la soluzione basata sul software libero invece può avere costi di manutenzione stimabili a priori poiché la manutenzione può essere fatta da chiunque.\\
Un ultimo costo di cui tenere conto è sicuramente il costo di personalizzazione dello strumento. Con un prodotto commerciale può capitare di non avere margini di personalizzazione, o comunque in caso di personalizzazione si rischiano costi solitamente molto elevati. Con una soluzione libera invece si profilano due scenari: il codice sorgente è open source, e quindi è liberamente scaricabile e modificabile anche a basso costo (direttamente propozionale al livello di training effettuato precedentemente, oppure al costo della consulenza esterna necessaria), oppure il codice sorgente non è open source e quindi ci si ritrova con una situazione molto simile a quella del software commerciale. \\
La personalizzazione è purtroppo qualcosa che non è possibile stimare a priori fino a quando non si è realizzata un'analisi dettagliata del package con cui si ha a che fare. \\
Un buon workflow per la scelta della soluzione adeguata può essere il seguente:
\begin{enumerate}
\item Identificare le soluzioni disponibili
\item Effettuare delle ricerche per capire come funziona ciascuna soluzione
\item Effettuare una comparazione secondo i seguenti parametri
\begin{itemize}
\item Funzionalità
\item Costi
\item Momentum
\item Supporto
\item Performance
\item Security
\item Usabilità
\item Licenza ed aspetti legali
\item Criteri specifici per l'ambiente di produzione
\end{itemize}
\item Effettuare un testing dettagliato
\item Conclusione e scelta
\end{enumerate}